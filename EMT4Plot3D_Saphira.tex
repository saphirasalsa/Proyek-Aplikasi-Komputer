\documentclass{article}

\usepackage{eumat}

\begin{document}
\begin{eulernotebook}
\begin{eulercomment}
Nama  : Saphira Nuria Salsabila\\
NIM   : 22305141050\\
Kelas : Matematika B\\
\end{eulercomment}
\eulersubheading{}
\eulerheading{Menggambar Plot 3D dengan EMT}
\begin{eulercomment}
Ini adalah pengenalan plot 3D di Euler. Kita memerlukan plot 3D untuk
memvisualisasikan fungsi dari dua variabel.

Euler menggambar fungsi-fungsi tersebut dengan menggunakan algoritme
pengurutan untuk menyembunyikan bagian-bagian di latar belakang.
Secara umum, Euler menggunakan proyeksi pusat. Standarnya adalah dari
kuadran x-y positif ke arah asal x=y=z=0, tetapi sudut=0° terlihat
dari arah sumbu-y. Sudut pandang dan ketinggian dapat diubah.

Euler dapat mem-plot :

- memplot - permukaan dengan garis bayangan dan garis datar,\\
- awan titik,\\
- kurva parametrik,\\
- permukaan implisit.


Plot 3D suatu fungsi menggunakan plot3d. Cara termudah adalah dengan
memplot ekspresi dalam x dan y. Parameter r mengatur rentang plot
sekitar (0,0).
\end{eulercomment}
\begin{eulerprompt}
>aspect(1.5); plot3d("x^2+sin(y)",-5,5,0,6*pi):
\end{eulerprompt}
\eulerimg{17}{images/EMT4Plot3D_Saphira-001.png}
\begin{eulerprompt}
>plot3d("x^2+x*sin(y)",-5,5,0,6*pi):
\end{eulerprompt}
\eulerimg{17}{images/EMT4Plot3D_Saphira-002.png}
\begin{eulercomment}
Silakan lakukan modifikasi agar gambar "talang bergelombang" tersebut tidak lurus melainkan melengkung/melingkar, baik
melingkar secara mendatar maupun melingkar turun/naik (seperti papan peluncur pada kolam renang. Temukan rumusnya.
\end{eulercomment}
\begin{eulerprompt}
>aspect(1.5); plot3d("x^2+sin(y)",r=pi):
\end{eulerprompt}
\eulerimg{17}{images/EMT4Plot3D_Saphira-003.png}
\eulerheading{Fungsi dua Variabel}
\begin{eulercomment}
Untuk grafik suatu fungsi, gunakan 

- ekspresi sederhana dalam x dan y,\\
- nama fungsi dari dua variabel,\\
- atau matriks data.

Standarnya adalah kisi-kisi kawat berisi dengan warna berbeda di kedua
sisi. Perhatikan bahwa jumlah interval kisi default adalah 10, tetapi
plot menggunakan jumlah default persegi panjang 40x40 untuk membuat
permukaannya. Ini bisa diubah.

- n=40, n=[40,40]: jumlah garis grid di setiap arah.\\
- grid=10, grid=[10,10]: : jumlah garis grid di setiap arah.

Kita menggunakan default n=40 dan grid=10.
\end{eulercomment}
\begin{eulerprompt}
>plot3d("x^2+y^2"):
\end{eulerprompt}
\eulerimg{17}{images/EMT4Plot3D_Saphira-004.png}
\begin{eulercomment}
Interaksi pengguna dimungkinkan dengan parameter \textgreater{}pengguna. Pengguna
dapat menekan tombol berikut.

- left,right,up,down: : putar sudut pandang,\\
- +,-: zoom in or out\\
- a: menghasilkan anaglyph (lihat di bawah)\\
- l:  beralih memutar sumber cahaya(lihat di bawah)\\
- space: reset to default\\
- return: end interaction
\end{eulercomment}
\begin{eulerprompt}
>plot3d("exp(-x^2+y^2)",>user, ...
>  title="Turn with the vector keys (press return to finish)"):
\end{eulerprompt}
\eulerimg{17}{images/EMT4Plot3D_Saphira-005.png}
\begin{eulercomment}
Rentang plot untuk fungsi dapat ditentukan dengan :

- a,b: the x-range\\
- c,d: the y-range\\
- r: a symmetric square around (0,0).\\
- n: number of subintervals for the plot.

Ada beberapa parameter untuk menskalakan fungsi atau mengubah tampilan
grafik.

fscale: scales to function values (default is \textless{}fscale).\\
scale: angka atau vektor 1x2 untuk menskalakan ke arah x dan y\\
frame:  jenis bingkai (default 1).
\end{eulercomment}
\begin{eulerprompt}
>plot3d("exp(-(x^2+y^2)/5)",r=10,n=80,fscale=4,scale=1.2,frame=3,>user):
\end{eulerprompt}
\eulerimg{17}{images/EMT4Plot3D_Saphira-006.png}
\begin{eulercomment}
Tampilan dapat diubah dengan berbagai cara.

- distance: jarak pandang ke plot.\\
- zoom: the zoom value.\\
- sudut: the angle to the negative y-axis in radians.\\
- height: the height of the view in radians.

Nilai default dapat diperiksa atau diubah dengan fungsi view(). Ini
mengembalikan parameter dalam urutan di atas.
\end{eulercomment}
\begin{eulerprompt}
>view
\end{eulerprompt}
\begin{euleroutput}
  [5,  2.6,  2,  0.4]
\end{euleroutput}
\begin{eulercomment}
Jarak yang lebih dekat membutuhkan lebih sedikit zoom. Efeknya lebih
seperti lensa sudut lebar.

ada contoh berikut, sudut=0 dan tinggi=0 dilihat dari sumbu y negatif.
Label sumbu untuk y disembunyikan dalam kasus ini.
\end{eulercomment}
\begin{eulerprompt}
>plot3d("x^2+y",distance=3,zoom=1,angle=pi/2,height=0):
\end{eulerprompt}
\eulerimg{17}{images/EMT4Plot3D_Saphira-007.png}
\begin{eulercomment}
Plot selalu terlihat berada di tengah kubus plot. Anda dapat
memindahkan bagian tengah dengan parameter tengah.
\end{eulercomment}
\begin{eulerprompt}
>plot3d("x^4+y^2",a=0,b=1,c=-1,d=1,angle=-20°,height=20°, ...
>  center=[0.4,0,0],zoom=5):
\end{eulerprompt}
\eulerimg{17}{images/EMT4Plot3D_Saphira-008.png}
\begin{eulercomment}
Plotnya diskalakan agar sesuai dengan unit kubus untuk dilihat. Jadi
tidak perlu mengubah jarak atau zoom tergantung ukuran plot. Namun
labelnya mengacu pada ukuran sebenarnya.

Jika Anda mematikannya dengan scale=false, Anda harus berhati-hati
agar plot tetap masuk ke dalam jendela plotting, dengan mengubah jarak
pandang atau zoom, dan memindahkan bagian tengah.
\end{eulercomment}
\begin{eulerprompt}
>plot3d("5*exp(-x^2-y^2)",r=2,<fscale,<scale,distance=13,height=50°, ...
>  center=[0,0,-2],frame=3):
\end{eulerprompt}
\eulerimg{17}{images/EMT4Plot3D_Saphira-009.png}
\begin{eulercomment}
Plot kutub juga tersedia. Parameter polar=true menggambar plot kutub.
Fungsi tersebut harus tetap merupakan fungsi dari x dan y. Parameter
"fscale" menskalakan fungsi dengan skalanya sendiri. Kalau tidak,
fungsinya akan diskalakan agar sesuai dengan kubus.
\end{eulercomment}
\begin{eulerprompt}
>plot3d("1/(x^2+y^2+1)",r=5,>polar, ...
>fscale=2,>hue,n=100,zoom=4,>contour,color=blue):
\end{eulerprompt}
\eulerimg{17}{images/EMT4Plot3D_Saphira-010.png}
\begin{eulerprompt}
>function f(r) := exp(-r/2)*cos(r); ...
>plot3d("f(x^2+y^2)",>polar,scale=[1,1,0.4],r=pi,frame=3,zoom=4):
\end{eulerprompt}
\eulerimg{17}{images/EMT4Plot3D_Saphira-011.png}
\begin{eulercomment}
Parameter memutar memutar fungsi di x di sekitar sumbu x.

- rotate=1: Uses the x-axis\\
- rotate=2: Uses the z-axis
\end{eulercomment}
\begin{eulerprompt}
>plot3d("x^2+1",a=-1,b=1,rotate=true,grid=5):
\end{eulerprompt}
\eulerimg{17}{images/EMT4Plot3D_Saphira-012.png}
\begin{eulerprompt}
>plot3d("x^2+1",a=-1,b=1,rotate=2,grid=5):
\end{eulerprompt}
\eulerimg{17}{images/EMT4Plot3D_Saphira-013.png}
\begin{eulerprompt}
>plot3d("sqrt(25-x^2)",a=0,b=5,rotate=1):
\end{eulerprompt}
\eulerimg{17}{images/EMT4Plot3D_Saphira-014.png}
\begin{eulerprompt}
>plot3d("x*sin(x)",a=0,b=6pi,rotate=2):
\end{eulerprompt}
\eulerimg{17}{images/EMT4Plot3D_Saphira-015.png}
\begin{eulercomment}
Berikut adalah plot dengan tiga fungsi.
\end{eulercomment}
\begin{eulerprompt}
>plot3d("x","x^2+y^2","y",r=2,zoom=3.5,frame=3):
\end{eulerprompt}
\eulerimg{17}{images/EMT4Plot3D_Saphira-016.png}
\eulerheading{Plot Kontur}
\begin{eulercomment}
Untuk plotnya, Euler menambahkan garis grid. Sebaliknya dimungkinkan
untuk menggunakan garis datar dan rona satu warna atau rona warna
spektral. Euler dapat menggambar ketinggian fungsi pada plot dengan
arsiran. Di semua plot 3D, Euler dapat menghasilkan anaglyph.

- \textgreater{}hue:  Mengaktifkan bayangan cahaya, bukan kabel.

\end{eulercomment}
\begin{eulerttcomment}
 >contour: : Membuat plot garis kontur otomatis pada plot.
\end{eulerttcomment}
\begin{eulercomment}
- level=... (or levels): A Vektor nilai garis kontur.

Standarnya adalah level="auto", yang menghitung beberapa garis level
secara otomatis. Seperti yang Anda lihat di plot, level sebenarnya
adalah rentang level.

Gaya default dapat diubah. Untuk plot kontur berikut, kami menggunakan
grid yang lebih halus berukuran 100x100 poin, menskalakan fungsi dan
plot, dan menggunakan sudut pandang yang berbeda.
\end{eulercomment}
\begin{eulerprompt}
>plot3d("exp(-x^2-y^2)",r=2,n=100,level="thin", ...
> >contour,>spectral,fscale=1,scale=1.1,angle=45°,height=20°):
\end{eulerprompt}
\eulerimg{17}{images/EMT4Plot3D_Saphira-017.png}
\begin{eulerprompt}
>plot3d("exp(x*y)",angle=100°,>contour,color=green):
\end{eulerprompt}
\eulerimg{17}{images/EMT4Plot3D_Saphira-018.png}
\begin{eulercomment}
Bayangan defaultnya menggunakan warna abu-abu. Namun rentang warna
spektral juga tersedia. \\
- \textgreater{}spectral: Menggunakan skema spektral default\\
- color=...: Menggunakan warna khusus atau skema

spektral Untuk plot berikut, kami menggunakan skema spektral default
dan menambah jumlah titik untuk mendapatkan tampilan yang sangat
halus.
\end{eulercomment}
\begin{eulerprompt}
>plot3d("x^2+y^2",>spectral,>contour,n=100):
\end{eulerprompt}
\eulerimg{17}{images/EMT4Plot3D_Saphira-019.png}
\begin{eulercomment}
Selain garis level otomatis, kita juga dapat menetapkan nilai garis
level. Ini akan menghasilkan garis level yang tipis, bukan rentang
level.
\end{eulercomment}
\begin{eulerprompt}
>plot3d("x^2-y^2",0,5,0,5,level=-1:0.1:1,color=redgreen):
\end{eulerprompt}
\eulerimg{17}{images/EMT4Plot3D_Saphira-020.png}
\begin{eulercomment}
Dalam plot berikut, kita menggunakan dua pita tingkat yang sangat luas
dari -0,1 hingga 1, dan dari 0,9 hingga 1. Ini dimasukkan sebagai
matriks dengan batas tingkat sebagai kolom.

Selain itu, kami melapisi grid dengan 10 interval di setiap arah.
\end{eulercomment}
\begin{eulerprompt}
>plot3d("x^2+y^3",level=[-0.1,0.9;0,1], ...
>  >spectral,angle=30°,grid=10,contourcolor=gray):
\end{eulerprompt}
\eulerimg{17}{images/EMT4Plot3D_Saphira-021.png}
\begin{eulercomment}
Pada contoh berikut, kita memplot himpunan, di mana :

\end{eulercomment}
\begin{eulerformula}
\[
f(x,y) = x^y-y^x = 0
\]
\end{eulerformula}
\begin{eulercomment}
Kami menggunakan satu garis tipis untuk garis level.
\end{eulercomment}
\begin{eulerprompt}
>plot3d("x^y-y^x",level=0,a=0,b=6,c=0,d=6,contourcolor=red,n=100):
\end{eulerprompt}
\eulerimg{17}{images/EMT4Plot3D_Saphira-022.png}
\begin{eulercomment}
Dimungkinkan untuk menampilkan bidang kontur di bawah plot. Warna dan
jarak ke plot dapat ditentukan.
\end{eulercomment}
\begin{eulerprompt}
>plot3d("x^2+y^4",>cp,cpcolor=green,cpdelta=0.2):
\end{eulerprompt}
\eulerimg{17}{images/EMT4Plot3D_Saphira-023.png}
\begin{eulercomment}
Berikut beberapa gaya lainnya. Kami selalu mematikan bingkai, dan
menggunakan berbagai skema warna untuk plot dan kisi.
\end{eulercomment}
\begin{eulerprompt}
>figure(2,2); ...
>expr="y^3-x^2"; ...
>figure(1);  ...
>  plot3d(expr,<frame,>cp,cpcolor=spectral); ...
>figure(2);  ...
>  plot3d(expr,<frame,>spectral,grid=10,cp=2); ...
>figure(3);  ...
>  plot3d(expr,<frame,>contour,color=gray,nc=5,cp=3,cpcolor=greenred); ...
>figure(4);  ...
>  plot3d(expr,<frame,>hue,grid=10,>transparent,>cp,cpcolor=gray); ...
>figure(0):
\end{eulerprompt}
\eulerimg{17}{images/EMT4Plot3D_Saphira-024.png}
\begin{eulercomment}
Ada beberapa skema spektral lainnya, yang diberi nomor dari 1 hingga
9. Namun Anda juga dapat menggunakan warna=nilai, di mana nilai :

- spectral: untuk rentang dari biru ke merah\\
- white: untuk rentang yang lebih redup \\
- yellowblue,purplegreen,blueyellow,greenred\\
- blueyellow, greenpurple,yellowblue,redgreen
\end{eulercomment}
\begin{eulerprompt}
>figure(3,3); ...
>for i=1:9;  ...
>  figure(i); plot3d("x^2+y^2",spectral=i,>contour,>cp,<frame,zoom=4);  ...
>end; ...
>figure(0):
\end{eulerprompt}
\eulerimg{17}{images/EMT4Plot3D_Saphira-025.png}
\begin{eulercomment}
Sumber cahaya dapat diubah dengan l dan tombol kursor selama interaksi
pengguna. Itu juga dapat diatur dengan parameter.

- light: arah\\
- amb: ambient light between 0 and 1

Catatan : program tidak membuat perbedaan antara sisi plot. Tidak ada
bayangan. Untuk ini, Anda memerlukan Povray.
\end{eulercomment}
\begin{eulerprompt}
>plot3d("-x^2-y^2", ...
>  hue=true,light=[0,1,1],amb=0,user=true, ...
>  title="Press l and cursor keys (return to exit)"):
\end{eulerprompt}
\eulerimg{17}{images/EMT4Plot3D_Saphira-026.png}
\begin{eulercomment}
Parameter warna mengubah warna permukaan. Warna garis level juga bisa
diubah.
\end{eulercomment}
\begin{eulerprompt}
>plot3d("-x^2-y^2",color=rgb(0.2,0.2,0),hue=true,frame=false, ...
>  zoom=3,contourcolor=red,level=-2:0.1:1,dl=0.01):
\end{eulerprompt}
\eulerimg{17}{images/EMT4Plot3D_Saphira-027.png}
\begin{eulercomment}
The color 0 gives a special rainbow effect.
\end{eulercomment}
\begin{eulerprompt}
>plot3d("x^2/(x^2+y^2+1)",color=0,hue=true,grid=10):
\end{eulerprompt}
\eulerimg{17}{images/EMT4Plot3D_Saphira-028.png}
\begin{eulercomment}
The surface can also be transparent.
\end{eulercomment}
\begin{eulerprompt}
>plot3d("x^2+y^2",>transparent,grid=10,wirecolor=red):
\end{eulerprompt}
\eulerimg{17}{images/EMT4Plot3D_Saphira-029.png}
\eulerheading{Plot Implisit}
\begin{eulercomment}
Ada juga plot implisit dalam tiga dimensi. Euler menghasilkan
pemotongan melalui objek. Fitur plot3d mencakup plot implisit. Plot
ini menunjukkan himpunan nol suatu fungsi dalam tiga variabel.
Permukaannya juga bisa transparan.\\
Solusi dari


atex: f(x,y,z) = 0

dapat divisualisasikan dalam potongan yang sejajar dengan bidang xy-,
xz- dan yz.

- implicit=1: cut parallel to the y-z-plane\\
- implicit=2: cut parallel to the x-z-plane\\
- implicit=4: cut parallel to the x-y-plane

Tambahkan nilai berikut, jika Anda mau. Dalam contoh kita memplot :

\end{eulercomment}
\begin{eulerformula}
\[
M = \{ (x,y,z) : x^2+y^3+zy=1 \}
\]
\end{eulerformula}
\begin{eulerprompt}
>plot3d("x^2+y^3+z*y-1",r=5,implicit=3):
\end{eulerprompt}
\eulerimg{17}{images/EMT4Plot3D_Saphira-030.png}
\begin{eulerprompt}
>c=1; d=1;
>plot3d("((x^2+y^2-c^2)^2+(z^2-1)^2)*((y^2+z^2-c^2)^2+(x^2-1)^2)*((z^2+x^2-c^2)^2+(y^2-1)^2)-d",r=2,<frame,>implicit,>user): 
\end{eulerprompt}
\eulerimg{17}{images/EMT4Plot3D_Saphira-031.png}
\begin{euleroutput}
  Cannot combine a 41x41 and a 1x81 matrix for +!
  Error in expression: ((x^2+y^2-c^2)^2+(z^2-1)^2)*((y^2+z^2-c^2)^2+(x^2-1)^2)*((z^2+x^2-c^2)^2+(y^2-1)^2)-d
  Try "trace errors" to inspect local variables after errors.
  pov3d:
      z=f(x,y;args());
\end{euleroutput}
\begin{eulerprompt}
>plot3d("x^2+y^2+4*x*z+z^3",>implicit,r=2,zoom=2.5):
\end{eulerprompt}
\eulerimg{17}{images/EMT4Plot3D_Saphira-032.png}
\eulerheading{Merencanakan Data 3D}
\begin{eulercomment}
Sama seperti plot2d, plot3d menerima data. Untuk objek 3D, Anda perlu
menyediakan matriks nilai x-, y- dan z, atau tiga fungsi atau ekspresi
fx(x,y), fy(x,y), fz(x,y).

\end{eulercomment}
\begin{eulerformula}
\[
\gamma(t,s) = (x(t,s),y(t,s),z(t,s))
\]
\end{eulerformula}
\begin{eulercomment}
Karena x,y,z adalah matriks, kita asumsikan bahwa (t,s) melewati grid
persegi. Hasilnya, Anda dapat memplot gambar persegi panjang di ruang
angkasa.\\
Anda dapat menggunakan bahasa matriks Euler untuk menghasilkan
koordinat secara efektif.

Dalam contoh berikut, kita menggunakan vektor nilai t dan vektor kolom
nilai s untuk membuat parameter permukaan bola. Dalam gambar kita
dapat menandai wilayah, dalam kasus kita wilayah kutub.
\end{eulercomment}
\begin{eulerprompt}
>t=linspace(0,2pi,180); s=linspace(-pi/2,pi/2,90)'; ...
>x=cos(s)*cos(t); y=cos(s)*sin(t); z=sin(s); ...
>plot3d(x,y,z,>hue, ...
>color=blue,<frame,grid=[10,20], ...
>values=s,contourcolor=red,level=[90°-24°;90°-22°], ...
>scale=1.4,height=50°):
\end{eulerprompt}
\eulerimg{17}{images/EMT4Plot3D_Saphira-034.png}
\begin{eulercomment}
Here is an example, which is the graph of a function.
\end{eulercomment}
\begin{eulerprompt}
>t=-1:0.1:1; s=(-1:0.1:1)'; plot3d(t,s,t*s,grid=10):
\end{eulerprompt}
\eulerimg{17}{images/EMT4Plot3D_Saphira-035.png}
\begin{eulercomment}
Namun, kita bisa membuat berbagai macam permukaan. Berikut adalah
permukaan yang sama sebagai suatu fungsi :

\end{eulercomment}
\begin{eulerformula}
\[
x = y \, z
\]
\end{eulerformula}
\begin{eulerprompt}
>plot3d(t*s,t,s,angle=180°,grid=10):
\end{eulerprompt}
\eulerimg{17}{images/EMT4Plot3D_Saphira-036.png}
\begin{eulercomment}
Dengan lebih banyak usaha, kita dapat menghasilkan banyak permukaan.


alam contoh berikut kita membuat tampilan bayangan dari bola yang
terdistorsi. Koordinat bola yang biasa adalah

\end{eulercomment}
\begin{eulerformula}
\[
\gamma(t,s) = (\cos(t)\cos(s),\sin(t)\sin(s),\cos(s))
\]
\end{eulerformula}
\begin{eulercomment}
with

\end{eulercomment}
\begin{eulerformula}
\[
0 \le t \le 2\pi, \quad \frac{-\pi}{2} \le s \le \frac{\pi}{2}.
\]
\end{eulerformula}
\begin{eulercomment}
Kami mendistorsi ini dengan sebuah faktor

\end{eulercomment}
\begin{eulerformula}
\[
d(t,s) = \frac{\cos(4t)+\cos(8s)}{4}.
\]
\end{eulerformula}
\begin{eulerprompt}
>t=linspace(0,2pi,320); s=linspace(-pi/2,pi/2,160)'; ...
>d=1+0.2*(cos(4*t)+cos(8*s)); ...
>plot3d(cos(t)*cos(s)*d,sin(t)*cos(s)*d,sin(s)*d,hue=1, ...
>  light=[1,0,1],frame=0,zoom=5):
\end{eulerprompt}
\eulerimg{17}{images/EMT4Plot3D_Saphira-037.png}
\begin{eulercomment}
Tentu saja, point cloud juga dimungkinkan. Untuk memplot data titik
dalam ruang, kita memerlukan tiga vektor untuk koordinat titik-titik
tersebut.

Gayanya sama seperti di plot2d dengan points=true;
\end{eulercomment}
\begin{eulerprompt}
>n=500;  ...
>  plot3d(normal(1,n),normal(1,n),normal(1,n),points=true,style="."):
\end{eulerprompt}
\eulerimg{17}{images/EMT4Plot3D_Saphira-038.png}
\begin{eulercomment}
Dimungkinkan juga untuk memplot kurva dalam 3D. Dalam hal ini, lebih
mudah untuk menghitung terlebih dahulu titik-titik kurva. Untuk kurva
pada bidang kita menggunakan barisan koordinat dan parameter
wire=true.
\end{eulercomment}
\begin{eulerprompt}
>t=linspace(0,8pi,500); ...
>plot3d(sin(t),cos(t),t/10,>wire,zoom=3):
\end{eulerprompt}
\eulerimg{17}{images/EMT4Plot3D_Saphira-039.png}
\begin{eulerprompt}
>t=linspace(0,4pi,1000); plot3d(cos(t),sin(t),t/2pi,>wire, ...
>linewidth=3,wirecolor=blue):
\end{eulerprompt}
\eulerimg{17}{images/EMT4Plot3D_Saphira-040.png}
\begin{eulerprompt}
>X=cumsum(normal(3,100)); ...
> plot3d(X[1],X[2],X[3],>anaglyph,>wire):
\end{eulerprompt}
\eulerimg{17}{images/EMT4Plot3D_Saphira-041.png}
\begin{eulercomment}
EMT juga dapat membuat plot dalam mode anaglyph. Untuk melihat plot
seperti itu, Anda memerlukan kacamata berwarna merah/cyan.
\end{eulercomment}
\begin{eulerprompt}
> plot3d("x^2+y^3",>anaglyph,>contour,angle=30°):
\end{eulerprompt}
\eulerimg{17}{images/EMT4Plot3D_Saphira-042.png}
\begin{eulercomment}
Seringkali skema warna spektral digunakan untuk plot. Ini menekankan
ketinggian fungsinya.
\end{eulercomment}
\begin{eulerprompt}
>plot3d("x^2*y^3-y",>spectral,>contour,zoom=3.2):
\end{eulerprompt}
\eulerimg{17}{images/EMT4Plot3D_Saphira-043.png}
\begin{eulercomment}
Euler juga dapat memplot permukaan yang diparameterisasi, jika
parameternya adalah nilai x, y, dan z dari gambar kotak persegi
panjang di ruang tersebut.

Untuk demo berikut, kami menyiapkan parameter u- dan v-, dan
menghasilkan koordinat ruang dari parameter tersebut.

\end{eulercomment}
\begin{eulerttcomment}
 t \(\backslash\)le 2\(\backslash\)pi, \(\backslash\)quad \(\backslash\)frac\{-\(\backslash\)pi\}\{2\} \(\backslash\)le s \(\backslash\)le \(\backslash\)frac\{\(\backslash\)pi\}\{2\}.
\end{eulerttcomment}
\begin{eulerprompt}
>u=linspace(-1,1,10); v=linspace(0,2*pi,50)'; ...
>X=(3+u*cos(v/2))*cos(v); Y=(3+u*cos(v/2))*sin(v); Z=u*sin(v/2); ...
>plot3d(X,Y,Z,>anaglyph,<frame,>wire,scale=2.3):
\end{eulerprompt}
\eulerimg{17}{images/EMT4Plot3D_Saphira-044.png}
\begin{eulercomment}
Here is a more complicated example, which is majestic with red/cyan glasses.
\end{eulercomment}
\begin{eulerprompt}
>u:=linspace(-pi,pi,160); v:=linspace(-pi,pi,400)';  ...
>x:=(4*(1+.25*sin(3*v))+cos(u))*cos(2*v); ...
>y:=(4*(1+.25*sin(3*v))+cos(u))*sin(2*v); ...
> z=sin(u)+2*cos(3*v); ...
>plot3d(x,y,z,frame=0,scale=1.5,hue=1,light=[1,0,-1],zoom=2.8,>anaglyph):
\end{eulerprompt}
\eulerimg{17}{images/EMT4Plot3D_Saphira-045.png}
\eulerheading{Plot Statistik}
\begin{eulercomment}
Plot batang juga dimungkinkan. Untuk ini, kita harus menyediakan

- x: row vector with n+1 elements\\
- y: column vector with n+1 elements\\
- z: nxn matrix of values.

z can be larger, but only nxn values will be used.

Dalam contoh ini, pertama-tama kita menghitung nilainya. Kemudian kita
sesuaikan x dan y, sehingga vektor-vektornya berpusat pada nilai yang
digunakan.
\end{eulercomment}
\begin{eulerprompt}
>x=-1:0.1:1; y=x'; z=x^2+y^2; ...
>xa=(x|1.1)-0.05; ya=(y_1.1)-0.05; ...
>plot3d(xa,ya,z,bar=true):
\end{eulerprompt}
\eulerimg{17}{images/EMT4Plot3D_Saphira-046.png}
\begin{eulercomment}
Dimungkinkan untuk membagi plot suatu permukaan menjadi dua bagian
atau lebih.
\end{eulercomment}
\begin{eulerprompt}
>x=-1:0.1:1; y=x'; z=x+y; d=zeros(size(x)); ...
>plot3d(x,y,z,disconnect=2:2:20):
\end{eulerprompt}
\eulerimg{17}{images/EMT4Plot3D_Saphira-047.png}
\begin{eulercomment}
Jika memuat atau menghasilkan matriks data M dari file dan perlu
memplotnya dalam 3D, Anda dapat menskalakan matriks ke [-1,1] dengan
skala(M), atau menskalakan matriks dengan \textgreater{}zscale. Hal ini dapat
dikombinasikan dengan faktor penskalaan individual yang diterapkan
sebagai tambahan.
\end{eulercomment}
\begin{eulerprompt}
>i=1:20; j=i'; ...
>plot3d(i*j^2+100*normal(20,20),>zscale,scale=[1,1,1.5],angle=-40°,zoom=1.8):
\end{eulerprompt}
\eulerimg{17}{images/EMT4Plot3D_Saphira-048.png}
\begin{eulerprompt}
>Z=intrandom(5,100,6); v=zeros(5,6); ...
>loop 1 to 5; v[#]=getmultiplicities(1:6,Z[#]); end; ...
>columnsplot3d(v',scols=1:5,ccols=[1:5]):
\end{eulerprompt}
\eulerimg{17}{images/EMT4Plot3D_Saphira-049.png}
\eulerheading{Permukaan Benda Putar}
\begin{eulerprompt}
>plot2d("(x^2+y^2-1)^3-x^2*y^3",r=1.3, ...
>style="#",color=red,<outline, ...
>level=[-2;0],n=100):
\end{eulerprompt}
\eulerimg{17}{images/EMT4Plot3D_Saphira-050.png}
\begin{eulerprompt}
>ekspresi &= (x^2+y^2-1)^3-x^2*y^3; $ekspresi
\end{eulerprompt}
\begin{eulerformula}
\[
\left(y^2+x^2-1\right)^3-x^2\,y^3
\]
\end{eulerformula}
\begin{eulercomment}
Kami ingin memutar kurva hati di sekitar sumbu y. Inilah ungkapan yang
mendefinisikan hati:

\end{eulercomment}
\begin{eulerformula}
\[
f(x,y)=(x^2+y^2-1)^3-x^2.y^3.
\]
\end{eulerformula}
\begin{eulercomment}
Next we set

\end{eulercomment}
\begin{eulerformula}
\[
x=r.cos(a),\quad y=r.sin(a).
\]
\end{eulerformula}
\begin{eulerprompt}
>function fr(r,a) &= ekspresi with [x=r*cos(a),y=r*sin(a)] | trigreduce; $fr(r,a)
\end{eulerprompt}
\begin{eulerformula}
\[
\left(r^2-1\right)^3+\frac{\left(\sin \left(5\,a\right)-\sin \left(
 3\,a\right)-2\,\sin a\right)\,r^5}{16}
\]
\end{eulerformula}
\begin{eulercomment}
Hal ini memungkinkan untuk mendefinisikan fungsi numerik, yang
menyelesaikan r, jika a diberikan. Dengan fungsi tersebut kita dapat
memplot jantung yang diputar sebagai permukaan parametrik.
\end{eulercomment}
\begin{eulerprompt}
>function map f(a) := bisect("fr",0,2;a); ...
>t=linspace(-pi/2,pi/2,100); r=f(t);  ...
>s=linspace(pi,2pi,100)'; ...
>plot3d(r*cos(t)*sin(s),r*cos(t)*cos(s),r*sin(t), ...
>>hue,<frame,color=red,zoom=4,amb=0,max=0.7,grid=12,height=50°):
\end{eulerprompt}
\eulerimg{17}{images/EMT4Plot3D_Saphira-053.png}
\begin{eulercomment}
Berikut ini adalah plot 3D dari gambar di atas yang diputar
mengelilingi sumbu z. Kami mendefinisikan fungsi yang mendeskripsikan
objek.
\end{eulercomment}
\begin{eulerprompt}
>function f(x,y,z) ...
\end{eulerprompt}
\begin{eulerudf}
  r=x^2+y^2;
  return (r+z^2-1)^3-r*z^3;
   endfunction
\end{eulerudf}
\begin{eulerprompt}
>plot3d("f(x,y,z)", ...
>xmin=0,xmax=1.2,ymin=-1.2,ymax=1.2,zmin=-1.2,zmax=1.4, ...
>implicit=1,angle=-30°,zoom=2.5,n=[10,100,60],>anaglyph):
\end{eulerprompt}
\eulerimg{17}{images/EMT4Plot3D_Saphira-054.png}
\eulerheading{Plot 3D Khusus}
\begin{eulercomment}
Fungsi plot3d bagus untuk dimiliki, tetapi tidak memenuhi semua
kebutuhan. Selain rutinitas yang lebih mendasar, dimungkinkan untuk
mendapatkan plot berbingkai dari objek apa pun yang Anda suka.

Meskipun Euler bukan program 3D, ia dapat menggabungkan beberapa objek
dasar. Kami mencoba memvisualisasikan paraboloid dan garis
singgungnya.
\end{eulercomment}
\begin{eulerprompt}
>function myplot ...
\end{eulerprompt}
\begin{eulerudf}
    y=-1:0.01:1; x=(-1:0.01:1)';
    plot3d(x,y,0.2*(x-0.1)/2,<scale,<frame,>hue, ..
      hues=0.5,>contour,color=orange);
    h=holding(1);
    plot3d(x,y,(x^2+y^2)/2,<scale,<frame,>contour,>hue);
    holding(h);
  endfunction
\end{eulerudf}
\begin{eulercomment}
Now framedplot() provides the frames, and sets the views.
\end{eulercomment}
\begin{eulerprompt}
>framedplot("myplot",[-1,1,-1,1,0,1],height=0,angle=-30°, ...
>  center=[0,0,-0.7],zoom=3):
\end{eulerprompt}
\eulerimg{17}{images/EMT4Plot3D_Saphira-055.png}
\begin{eulercomment}
Dengan cara yang sama, Anda dapat memplot bidang kontur secara manual.
Perhatikan bahwa plot3d() menyetel jendela ke fullwindow(), secara
default, tetapi plotcontourplane() berasumsi demikian.
\end{eulercomment}
\begin{eulerprompt}
>x=-1:0.02:1.1; y=x'; z=x^2-y^4;
>function myplot (x,y,z) ...
\end{eulerprompt}
\begin{eulerudf}
    zoom(2);
    wi=fullwindow();
    plotcontourplane(x,y,z,level="auto",<scale);
    plot3d(x,y,z,>hue,<scale,>add,color=white,level="thin");
    window(wi);
    reset();
  endfunction
\end{eulerudf}
\begin{eulerprompt}
>myplot(x,y,z):
\end{eulerprompt}
\eulerimg{27}{images/EMT4Plot3D_Saphira-056.png}
\eulerheading{Animasi}
\begin{eulercomment}
Euler dapat menggunakan frame untuk melakukan pra-komputasi animasi.


alah satu fungsi yang memanfaatkan teknik ini adalah memutar. Itu
dapat mengubah sudut pandang dan menggambar ulang plot 3D. Fungsi ini
memanggil addpage() untuk setiap plot baru. Akhirnya ia menganimasikan
plotnya.


Silakan pelajari sumber rotasi untuk melihat lebih detail.
\end{eulercomment}
\begin{eulerprompt}
>function testplot () := plot3d("x^2+y^3"); ...
>rotate("testplot"); testplot():
\end{eulerprompt}
\eulerimg{27}{images/EMT4Plot3D_Saphira-057.png}
\eulerheading{Menggambar Povray}
\begin{eulercomment}
Dengan bantuan file Euler povray.e, Euler dapat menghasilkan file
Povray. Hasilnya sangat bagus untuk dilihat.

Anda perlu menginstal Povray (32bit atau 64bit) dari
http://www.povray.org/, dan meletakkan sub-direktori "bin" Povray ke jalur lingkungan, atau mengatur variabel "defaultpovray" dengan jalur lengkap yang mengarah ke "pvengine.exe".


ntarmuka Povray Euler menghasilkan file Povray di direktori home
pengguna, dan memanggil Povray untuk menguraikan file-file ini. Nama
file default adalah current.pov, dan direktori default adalah
eulerhome(), biasanya c:\textbackslash{}Users\textbackslash{}Username\textbackslash{}Euler. Povray menghasilkan
file PNG, yang dapat dimuat oleh Euler ke dalam notebook. Untuk
membersihkan file-file ini, gunakan povclear().


ungsi pov3d memiliki semangat yang sama dengan plot3d. Ini dapat
menghasilkan grafik fungsi f(x,y), atau permukaan dengan koordinat
X,Y,Z dalam matriks, termasuk garis level opsional. Fungsi ini memulai
raytracer secara otomatis, dan memuat adegan ke dalam notebook Euler.


elain pov3d(), ada banyak fungsi yang menghasilkan objek Povray.
Fungsi-fungsi ini mengembalikan string, yang berisi kode Povray untuk
objek. Untuk menggunakan fungsi ini, mulai file Povray dengan
povstart(). Kemudian gunakan writeln(...) untuk menulis objek ke file
adegan. Terakhir, akhiri file dengan povend(). Secara default,
raytracer akan dimulai, dan PNG akan dimasukkan ke dalam notebook
Euler.

Fungsi objek memiliki parameter yang disebut "tampilan", yang
memerlukan string dengan kode Povray untuk tekstur dan penyelesaian
objek. Fungsi povlook() dapat digunakan untuk menghasilkan string ini.
Ini memiliki parameter untuk warna, transparansi, Phong Shading dll.


Perhatikan bahwa alam semesta Povray memiliki sistem koordinat lain.
Antarmuka ini menerjemahkan semua koordinat ke sistem Povray. Jadi
Anda dapat terus berpikir dalam sistem koordinat Euler dengan z
menunjuk vertikal ke atas, dan sumbu x,y,z di tangan kanan. Fungsi
pov3d memiliki semangat yang sama dengan plot3d. Ini dapat
menghasilkan grafik fungsi f(x,y), atau permukaan dengan koordinat
X,Y,Z dalam matriks, termasuk garis level opsional. Fungsi ini memulai
raytracer secara otomatis, dan memuat adegan ke dalam notebook Euler.\\
Anda perlu memuat file povray
\end{eulercomment}
\begin{eulerprompt}
>load povray;
\end{eulerprompt}
\begin{eulercomment}
Pastikan, direktori Povray bin ada di jalurnya. Jika tidak, edit
variabel berikut sehingga berisi jalur ke povray yang dapat
dieksekusi.
\end{eulercomment}
\begin{eulerprompt}
>defaultpovray="C:\(\backslash\)Program Files\(\backslash\)POV-Ray\(\backslash\)v3.7\(\backslash\)bin\(\backslash\)pvengine.exe"
\end{eulerprompt}
\begin{euleroutput}
  C:\(\backslash\)Program Files\(\backslash\)POV-Ray\(\backslash\)v3.7\(\backslash\)bin\(\backslash\)pvengine.exe
\end{euleroutput}
\begin{eulercomment}
Untuk kesan pertama, kami memplot fungsi sederhana. Perintah berikut
menghasilkan file povray di direktori pengguna Anda, dan menjalankan
Povray untuk penelusuran sinar file ini.


ika Anda memulai perintah berikut, GUI Povray akan terbuka,
menjalankan file, dan menutup secara otomatis. Karena alasan keamanan,
Anda akan ditanya apakah Anda ingin mengizinkan file exe dijalankan.
Anda dapat menekan batal untuk menghentikan pertanyaan lebih lanjut.
Anda mungkin harus menekan OK di jendela Povray untuk mengonfirmasi
dialog pengaktifan Povray.
\end{eulercomment}
\begin{eulerprompt}
>plot3d("x^2+y^2",zoom=2):
>pov3d("x^2+y^2",zoom=3);
\end{eulerprompt}
\begin{eulercomment}
Kita dapat membuat fungsinya transparan dan menambahkan penyelesaian
lainnya. Kita juga dapat menambahkan garis level ke plot fungsi.
\end{eulercomment}
\begin{eulerprompt}
>pov3d("x^2+y^3",axiscolor=red,angle=-45°,>anaglyph, ...
>  look=povlook(cyan,0.2),level=-1:0.5:1,zoom=3.8);
\end{eulerprompt}
\begin{eulercomment}
Terkadang perlu untuk mencegah penskalaan fungsi, dan menskalakan
fungsi secara manual.

Kita memplot himpunan titik pada bidang kompleks, dimana hasil kali
jarak ke 1 dan -1 sama dengan 1.
\end{eulercomment}
\begin{eulerprompt}
>pov3d("((x-1)^2+y^2)*((x+1)^2+y^2)/40",r=2, ...
>  angle=-120°,level=1/40,dlevel=0.005,light=[-1,1,1],height=10°,n=50, ...
>  <fscale,zoom=3.8);
\end{eulerprompt}
\eulerheading{Merencanakan dengan Koordinat}
\begin{eulercomment}
Daripada menggunakan fungsi, kita bisa memplotnya dengan koordinat.
Seperti di plot3d, kita memerlukan tiga matriks untuk mendefinisikan
objek.

Dalam contoh ini kita memutar suatu fungsi di sekitar sumbu z.
\end{eulercomment}
\begin{eulerprompt}
>function f(x) := x^3-x+1; ...
>x=-1:0.01:1; t=linspace(0,2pi,50)'; ...
>Z=x; X=cos(t)*f(x); Y=sin(t)*f(x); ...
>pov3d(X,Y,Z,angle=40°,look=povlook(red,0.1),height=50°,axis=0,zoom=4,light=[10,5,15]);
\end{eulerprompt}
\begin{euleroutput}
  Function povlook not found.
  Try list ... to find functions!
  Error in:
  ... f(x); pov3d(X,Y,Z,angle=40°,look=povlook(red,0.1),height=50°,a ...
                                                       ^
\end{euleroutput}
\begin{eulercomment}
Pada contoh berikut, kita memplot gelombang teredam. Kami menghasilkan
gelombang dengan bahasa matriks Euler.

Kami juga menunjukkan, bagaimana objek tambahan dapat ditambahkan ke
adegan pov3d. Untuk pembuatan objek, lihat contoh berikut. Perhatikan
bahwa plot3d menskalakan plot, sehingga cocok dengan kubus satuan.
\end{eulercomment}
\begin{eulerprompt}
>r=linspace(0,1,80); phi=linspace(0,2pi,80)'; ...
>x=r*cos(phi); y=r*sin(phi); z=exp(-5*r)*cos(8*pi*r)/3;  ...
>pov3d(x,y,z,zoom=6,axis=0,height=30°,add=povsphere([0.5,0,0.25],0.15,povlook(red)), ...
>  w=500,h=300);
\end{eulerprompt}
\begin{euleroutput}
  exec:
      return _exec(program,param,dir,print,hidden,wait);
  povray:
      exec(program,params,defaulthome);
  Try "trace errors" to inspect local variables after errors.
  pov3d:
      if povray then povray(currentfile,w,h,w/h); endif;
\end{euleroutput}
\begin{eulercomment}
Dengan metode peneduh canggih Povray, sangat sedikit titik yang dapat
menghasilkan permukaan yang sangat halus. Hanya pada batas-batas dan
dalam bayangan, triknya mungkin terlihat jelas.

Untuk ini, kita perlu menjumlahkan vektor normal di setiap titik
matriks.
\end{eulercomment}
\begin{eulerprompt}
>Z &= x^2*y^3
\end{eulerprompt}
\begin{euleroutput}
  
                                   2  3
                                  x  y
  
\end{euleroutput}
\begin{eulercomment}
Persamaan permukaannya adalah [x,y,Z]. Kami menghitung dua turunan
dari x dan y dan mengambil perkalian silangnya sebagai normal.
\end{eulercomment}
\begin{eulerprompt}
>dx &= diff([x,y,Z],x); dy &= diff([x,y,Z],y);
\end{eulerprompt}
\begin{eulercomment}
Kami mendefinisikan normal sebagai produk silang dari turunan ini, dan
mendefinisikan fungsi koordinat
\end{eulercomment}
\begin{eulerprompt}
>N &= crossproduct(dx,dy); NX &= N[1]; NY &= N[2]; NZ &= N[3]; N,
\end{eulerprompt}
\begin{euleroutput}
  
                                 3       2  2
                         [- 2 x y , - 3 x  y , 1]
  
\end{euleroutput}
\begin{eulercomment}
We use only 25 points.
\end{eulercomment}
\begin{eulerprompt}
>x=-1:0.5:1; y=x';
>pov3d(x,y,Z(x,y),angle=10°, ...
>  xv=NX(x,y),yv=NY(x,y),zv=NZ(x,y),<shadow);
\end{eulerprompt}
\begin{eulercomment}
The following is the Trefoil knot done by A. Busser in Povray. There
is an improved version of this in the examples.

See: Examples\textbackslash{}Trefoil Knot \textbar{} Trefoil Knot

For a good look with not too many points, we add normal vectors here.
We use Maxima to compute the normals for us. First, the three
functions for the coordinates as symbolic expressions.
\end{eulercomment}
\begin{eulerprompt}
>X &= ((4+sin(3*y))+cos(x))*cos(2*y); ...
>Y &= ((4+sin(3*y))+cos(x))*sin(2*y); ...
>Z &= sin(x)+2*cos(3*y);
\end{eulerprompt}
\begin{eulercomment}
Then the two derivative vectors to x and y.
\end{eulercomment}
\begin{eulerprompt}
>dx &= diff([X,Y,Z],x); dy &= diff([X,Y,Z],y);
\end{eulerprompt}
\begin{eulercomment}
Now the normal, which is the cross product of the two derivatives.
\end{eulercomment}
\begin{eulerprompt}
>dn &= crossproduct(dx,dy);
\end{eulerprompt}
\begin{eulercomment}
We now evaluate all this numerically.
\end{eulercomment}
\begin{eulerprompt}
>x:=linspace(-%pi,%pi,40); y:=linspace(-%pi,%pi,100)';
\end{eulerprompt}
\begin{eulercomment}
The normal vectors are evaluations of the symbolic expressions dn[i]
for i=1,2,3. The syntax for this is \&"expression"(parameters). This is
an alternative to the method in the previous example, where we defined
symbolic expressions NX, NY, NZ first.
\end{eulercomment}
\begin{eulerprompt}
>pov3d(X(x,y),Y(x,y),Z(x,y),>anaglyph,axis=0,zoom=5,w=450,h=350, ...
>  <shadow,look=povlook(blue), ...
>  xv=&"dn[1]"(x,y), yv=&"dn[2]"(x,y), zv=&"dn[3]"(x,y));
\end{eulerprompt}
\begin{eulercomment}
We can also generate a grid in 3D.
\end{eulercomment}
\begin{eulerprompt}
>povstart(zoom=4); ...
>x=-1:0.5:1; r=1-(x+1)^2/6; ...
>t=(0°:30°:360°)'; y=r*cos(t); z=r*sin(t); ...
>writeln(povgrid(x,y,z,d=0.02,dballs=0.05)); ...
>povend();
\end{eulerprompt}
\begin{euleroutput}
  exec:
      return _exec(program,param,dir,print,hidden,wait);
  povray:
      exec(program,params,defaulthome);
  Try "trace errors" to inspect local variables after errors.
  povend:
      povray(file,w,h,aspect,exit); 
\end{euleroutput}
\begin{eulercomment}
With povgrid(), curves are possible.
\end{eulercomment}
\begin{eulerprompt}
>povstart(center=[0,0,1],zoom=3.6); ...
>t=linspace(0,2,1000); r=exp(-t); ...
>x=cos(2*pi*10*t)*r; y=sin(2*pi*10*t)*r; z=t; ...
>writeln(povgrid(x,y,z,povlook(red))); ...
>writeAxis(0,2,axis=3); ...
>povend();
\end{eulerprompt}
\eulerheading{Objek Povray}
\begin{eulercomment}
Di atas, kami menggunakan pov3d untuk memplot permukaan. Antarmuka
povray di Euler juga dapat menghasilkan objek Povray. Objek ini
disimpan sebagai string di Euler, dan perlu ditulis ke file Povray.

Kami memulai output dengan povstart().
\end{eulercomment}
\begin{eulerprompt}
>povstart(zoom=4);
\end{eulerprompt}
\begin{eulercomment}
Pertama kita mendefinisikan tiga silinder, dan menyimpannya dalam
string di Euler.

Fungsi povx() dll. hanya mengembalikan vektor [1,0,0], yang dapat
digunakan sebagai gantinya.
\end{eulercomment}
\begin{eulerprompt}
>c1=povcylinder(-povx,povx,1,povlook(red)); ...
>c2=povcylinder(-povy,povy,1,povlook(yellow)); ...
>c3=povcylinder(-povz,povz,1,povlook(blue)); ...
\end{eulerprompt}
\begin{eulercomment}
String tersebut berisi kode Povray, yang tidak perlu kita pahami pada
saat itu.

Fungsi povx() dll. hanya mengembalikan vektor [1,0,0], yang dapat
digunakan sebagai gantinya.
\end{eulercomment}
\begin{eulerprompt}
>c2
\end{eulerprompt}
\begin{euleroutput}
  cylinder \{ <0,0,-1>, <0,0,1>, 1
   texture \{ pigment \{ color rgb <0.941176,0.941176,0.392157> \}  \} 
   finish \{ ambient 0.2 \} 
   \}
\end{euleroutput}
\begin{eulercomment}
As you see, we added texture to the objects in three different colors.

Hal ini dilakukan oleh povlook(), yang mengembalikan string dengan
kode Povray yang relevan. Kita dapat menggunakan warna default Euler,
atau menentukan warna kita sendiri. Kita juga dapat menambahkan
transparansi, atau mengubah cahaya sekitar.
\end{eulercomment}
\begin{eulerprompt}
>povlook(rgb(0.1,0.2,0.3),0.1,0.5)
\end{eulerprompt}
\begin{euleroutput}
   texture \{ pigment \{ color rgbf <0.101961,0.2,0.301961,0.1> \}  \} 
   finish \{ ambient 0.5 \} 
  
\end{euleroutput}
\begin{eulercomment}
Sekarang kita mendefinisikan objek persimpangan, dan menulis hasilnya
ke file.\\
i dilakukan oleh povlook(), yang mengembalikan string dengan kode
Povray yang relevan. Kita dapat menggunakan warna default Euler, atau
menentukan warna kita sendiri. Kita juga dapat menambahkan
transparansi, atau mengubah cahaya sekitar.
\end{eulercomment}
\begin{eulerprompt}
>writeln(povintersection([c1,c2,c3]));
\end{eulerprompt}
\begin{eulercomment}
Persimpangan tiga silinder sulit untuk divisualisasikan jika Anda
belum pernah melihatnya sebelumnya.
\end{eulercomment}
\begin{eulerprompt}
>povend;
\end{eulerprompt}
\begin{eulercomment}
Fungsi berikut menghasilkan fraktal secara rekursif.

Fungsi pertama menunjukkan bagaimana Euler menangani objek Povray
sederhana. Fungsi povbox() mengembalikan string, yang berisi koordinat
kotak, tekstur, dan hasil akhir.
\end{eulercomment}
\begin{eulerprompt}
>function onebox(x,y,z,d) := povbox([x,y,z],[x+d,y+d,z+d],povlook());
>function fractal (x,y,z,h,n) ...
\end{eulerprompt}
\begin{eulerudf}
   if n==1 then writeln(onebox(x,y,z,h));
   else
     h=h/3;
     fractal(x,y,z,h,n-1);
     fractal(x+2*h,y,z,h,n-1);
     fractal(x,y+2*h,z,h,n-1);
     fractal(x,y,z+2*h,h,n-1);
     fractal(x+2*h,y+2*h,z,h,n-1);
     fractal(x+2*h,y,z+2*h,h,n-1);
     fractal(x,y+2*h,z+2*h,h,n-1);
     fractal(x+2*h,y+2*h,z+2*h,h,n-1);
     fractal(x+h,y+h,z+h,h,n-1);
   endif;
  endfunction
\end{eulerudf}
\begin{eulerprompt}
>povstart(fade=10,<shadow);
>fractal(-1,-1,-1,2,4);
>povend();
\end{eulerprompt}
\begin{eulercomment}
Perbedaan memungkinkan pemisahan satu objek dari objek lainnya.
Seperti persimpangan, ada bagian dari objek CSG di Povray.
\end{eulercomment}
\begin{eulerprompt}
>povstart(light=[5,-5,5],fade=10);
\end{eulerprompt}
\begin{eulercomment}
For this demonstration, we define an object in Povray, instead of
using a string in Euler. Definitions are written to the file
immediately.

A box coordinate of -1 just means [-1,-1,-1].
\end{eulercomment}
\begin{eulerprompt}
>povdefine("mycube",povbox(-1,1));
\end{eulerprompt}
\begin{eulercomment}
We can use this object in povobject(), which returns a string as
usual.
\end{eulercomment}
\begin{eulerprompt}
>c1=povobject("mycube",povlook(red));
\end{eulerprompt}
\begin{eulercomment}
We generate a second cube, and rotate and scale it a bit.
\end{eulercomment}
\begin{eulerprompt}
>c2=povobject("mycube",povlook(yellow),translate=[1,1,1], ...
>  rotate=xrotate(10°)+yrotate(10°), scale=1.2);
\end{eulerprompt}
\begin{eulercomment}
Then we take the difference of the two objects.
\end{eulercomment}
\begin{eulerprompt}
>writeln(povdifference(c1,c2));
\end{eulerprompt}
\begin{eulercomment}
Now add three axes.
\end{eulercomment}
\begin{eulerprompt}
>writeAxis(-1.2,1.2,axis=1); ...
>writeAxis(-1.2,1.2,axis=2); ...
>writeAxis(-1.2,1.2,axis=4); ...
>povend();
\end{eulerprompt}
\eulerheading{Fungsi Implisit}
\begin{eulercomment}
Povray dapat memplot himpunan di mana f(x,y,z)=0, seperti parameter
implisit di plot3d. Namun hasilnya terlihat jauh lebih baik.


intaks untuk fungsinya sedikit berbeda. Anda tidak dapat menggunakan
keluaran ekspresi Maxima atau Euler.


Latex:((x\textasciicircum{}2+y\textasciicircum{}2-c\textasciicircum{}2)\textasciicircum{}2+(z\textasciicircum{}2-1)\textasciicircum{}2)*((y\textasciicircum{}2+z\textasciicircum{}2-c\textasciicircum{}2)\textasciicircum{}2+(x\textasciicircum{}2-1)\textasciicircum{}2)*((z\textasciicircum{}2+x\textasciicircum{}2-c\textasciicircum{}2)\textasciicircum{}2+(y\textasciicircum{}2-1)\textasciicircum{}2)=d
\end{eulercomment}
\begin{eulerprompt}
>povstart(angle=70°,height=50°,zoom=4);
>c=0.1; d=0.1; ...
>writeln(povsurface("(pow(pow(x,2)+pow(y,2)-pow(c,2),2)+pow(pow(z,2)-1,2))*(pow(pow(y,2)+pow(z,2)-pow(c,2),2)+pow(pow(x,2)-1,2))*(pow(pow(z,2)+pow(x,2)-pow(c,2),2)+pow(pow(y,2)-1,2))-d",povlook(red))); ...
>povend();
\end{eulerprompt}
\begin{euleroutput}
  Error : Povray error!
  
  Error generated by error() command
  
  povray:
      error("Povray error!");
  Try "trace errors" to inspect local variables after errors.
  povend:
      povray(file,w,h,aspect,exit); 
\end{euleroutput}
\begin{eulerprompt}
>povstart(angle=25°,height=10°); 
>writeln(povsurface("pow(x,2)+pow(y,2)*pow(z,2)-1",povlook(blue),povbox(-2,2,"")));
>povend();
>povstart(angle=70°,height=50°,zoom=4);
\end{eulerprompt}
\begin{eulercomment}
Create the implicit surface. Note the different syntax in the
expression.
\end{eulercomment}
\begin{eulerprompt}
>writeln(povsurface("pow(x,2)*y-pow(y,3)-pow(z,2)",povlook(green))); ...
>writeAxes(); ...
>povend();
\end{eulerprompt}
\eulerheading{Objek Jaring}
\begin{eulercomment}
Dalam contoh ini, kami menunjukkan cara membuat objek mesh, dan
menggambarnya dengan informasi tambahan.

Kita ingin memaksimalkan xy pada kondisi x+y=1 dan mendemonstrasikan
sentuhan tangensial garis datar.
\end{eulercomment}
\begin{eulerprompt}
>povstart(angle=-10°,center=[0.5,0.5,0.5],zoom=7);
\end{eulerprompt}
\begin{eulercomment}
We cannot store the object in a string as before, since is too large. So we define the object in a Povray file using
#declare. The function povtriangle() does this automatically. It can accept normal vectors just like pov3d().

The following defines the mesh object, and writes it immediately into the file.
\end{eulercomment}
\begin{eulerprompt}
>x=0:0.02:1; y=x'; z=x*y; vx=-y; vy=-x; vz=1;
>mesh=povtriangles(x,y,z,"",vx,vy,vz);
\end{eulerprompt}
\begin{eulercomment}
Now we define two discs, which will be intersected with the surface.
\end{eulercomment}
\begin{eulerprompt}
>cl=povdisc([0.5,0.5,0],[1,1,0],2); ...
>ll=povdisc([0,0,1/4],[0,0,1],2);
\end{eulerprompt}
\begin{eulercomment}
Write the surface minus the two discs.
\end{eulercomment}
\begin{eulerprompt}
>writeln(povdifference(mesh,povunion([cl,ll]),povlook(green)));
\end{eulerprompt}
\begin{eulercomment}
Write the two intersections.
\end{eulercomment}
\begin{eulerprompt}
>writeln(povintersection([mesh,cl],povlook(red))); ...
>writeln(povintersection([mesh,ll],povlook(gray)));
\end{eulerprompt}
\begin{eulercomment}
Write a point at the maximum.
\end{eulercomment}
\begin{eulerprompt}
>writeln(povpoint([1/2,1/2,1/4],povlook(gray),size=2*defaultpointsize));
\end{eulerprompt}
\begin{eulercomment}
Add axes and finish.
\end{eulercomment}
\begin{eulerprompt}
>writeAxes(0,1,0,1,0,1,d=0.015); ...
>povend();
\end{eulerprompt}
\eulerheading{Anaglyphs di Povray}
\begin{eulercomment}
Untuk menghasilkan anaglyph untuk kacamata merah/cyan, Povray harus
dijalankan dua kali dari posisi kamera berbeda. Ini menghasilkan dua
file Povray dan dua file PNG, yang dimuat dengan fungsi
loadanaglyph().

entu saja, Anda memerlukan kacamata berwarna merah/cyan untuk melihat
contoh berikut dengan benar.

ungsi pov3d() memiliki saklar sederhana untuk menghasilkan anaglyph.
\end{eulercomment}
\begin{eulerprompt}
>pov3d("-exp(-x^2-y^2)/2",r=2,height=45°,>anaglyph, ...
>  center=[0,0,0.5],zoom=3.5);
\end{eulerprompt}
\begin{eulercomment}
If you create a scene with objects, you need to put the generation of
the scene into a function, and run it twice with different values for
the anaglyph parameter.
\end{eulercomment}
\begin{eulerprompt}
>function myscene ...
\end{eulerprompt}
\begin{eulerudf}
    s=povsphere(povc,1);
    cl=povcylinder(-povz,povz,0.5);
    clx=povobject(cl,rotate=xrotate(90°));
    cly=povobject(cl,rotate=yrotate(90°));
    c=povbox([-1,-1,0],1);
    un=povunion([cl,clx,cly,c]);
    obj=povdifference(s,un,povlook(red));
    writeln(obj);
    writeAxes();
  endfunction
\end{eulerudf}
\begin{eulercomment}
The function povanaglyph() does all this. The parameters are like in
povstart() and povend() combined.
\end{eulercomment}
\begin{eulerprompt}
>povanaglyph("myscene",zoom=4.5);
\end{eulerprompt}
\eulerheading{Mendefinisikan Objek sendiri}
\begin{eulercomment}
ntarmuka povray Euler berisi banyak objek. Namun Anda tidak dibatasi
pada hal ini. Anda dapat membuat objek sendiri, yang menggabungkan
objek lain, atau merupakan objek yang benar-benar baru.

Kami mendemonstrasikan torus. Perintah Povray untuk ini adalah
"torus". Jadi kami mengembalikan string dengan perintah ini dan
parameternya. Perhatikan bahwa torus selalu berpusat pada titik asal.
\end{eulercomment}
\begin{eulerprompt}
>function povdonat (r1,r2,look="") ...
\end{eulerprompt}
\begin{eulerudf}
    return "torus \{"+r1+","+r2+look+"\}";
  endfunction
\end{eulerudf}
\begin{eulercomment}
Here is our first torus.
\end{eulercomment}
\begin{eulerprompt}
>t1=povdonat(0.8,0.2)
\end{eulerprompt}
\begin{euleroutput}
  torus \{0.8,0.2\}
\end{euleroutput}
\begin{eulercomment}
Let us use this object to create a second torus, translated and
rotated.
\end{eulercomment}
\begin{eulerprompt}
>t2=povobject(t1,rotate=xrotate(90°),translate=[0.8,0,0])
\end{eulerprompt}
\begin{euleroutput}
  object \{ torus \{0.8,0.2\}
   rotate 90 *x 
   translate <0.8,0,0>
   \}
\end{euleroutput}
\begin{eulercomment}
Now we place these objects into a scene. For the look, we use Phong
Shading.
\end{eulercomment}
\begin{eulerprompt}
>povstart(center=[0.4,0,0],angle=0°,zoom=3.8,aspect=1.5); ...
>writeln(povobject(t1,povlook(green,phong=1))); ...
>writeln(povobject(t2,povlook(green,phong=1))); ...
\end{eulerprompt}
\begin{eulerttcomment}
 >povend();
\end{eulerttcomment}
\begin{eulercomment}
calls the Povray program. However, in case of errors, it does not
display the error. You should therefore use

\end{eulercomment}
\begin{eulerttcomment}
 >povend(<exit);
\end{eulerttcomment}
\begin{eulercomment}

if anything did not work. This will leave the Povray window open.
\end{eulercomment}
\begin{eulerprompt}
>povend(h=320,w=480);
\end{eulerprompt}
\begin{euleroutput}
  Function povstart not found.
  Try list ... to find functions!
  Error in:
  povstart(center=[0.4,0,0],angle=0°,zoom=3.8,aspect=1.5); writeln(povobject(t1,povlook(green,phong=1))); writeln(povobj ...
                                                         ^
\end{euleroutput}
\begin{eulercomment}
Here is a more elaborate example. We solve

\end{eulercomment}
\begin{eulerformula}
\[
Ax \le b, \quad x \ge 0, \quad c.x \to \text{Max.}
\]
\end{eulerformula}
\begin{eulercomment}
and show the feasible points and the optimum in a 3D plot.
\end{eulercomment}
\begin{eulerprompt}
>A=[10,8,4;5,6,8;6,3,2;9,5,6];
>b=[10,10,10,10]';
>c=[1,1,1];
\end{eulerprompt}
\begin{eulercomment}
First, let us check, if this example has a solution at all.
\end{eulercomment}
\begin{eulerprompt}
>x=simplex(A,b,c,>max,>check)'
\end{eulerprompt}
\begin{euleroutput}
  [0,  1,  0.5]
\end{euleroutput}
\begin{eulercomment}
Yes, it has.

Next we define two objects. The first is the plane

\end{eulercomment}
\begin{eulerformula}
\[
a \cdot x \le b
\]
\end{eulerformula}
\begin{eulerprompt}
>function oneplane (a,b,look="") ...
\end{eulerprompt}
\begin{eulerudf}
    return povplane(a,b,look)
  endfunction
\end{eulerudf}
\begin{eulercomment}
Then we define the intersection of all half spaces and a cube.
\end{eulercomment}
\begin{eulerprompt}
>function adm (A, b, r, look="") ...
\end{eulerprompt}
\begin{eulerudf}
    ol=[];
    loop 1 to rows(A); ol=ol|oneplane(A[#],b[#]); end;
    ol=ol|povbox([0,0,0],[r,r,r]);
    return povintersection(ol,look);
  endfunction
\end{eulerudf}
\begin{eulercomment}
We can now plot the scene.
\end{eulercomment}
\begin{eulerprompt}
>povstart(angle=120°,center=[0.5,0.5,0.5],zoom=3.5); ...
>writeln(adm(A,b,2,povlook(green,0.4))); ...
>writeAxes(0,1.3,0,1.6,0,1.5); ...
\end{eulerprompt}
\begin{eulercomment}
The following is a circle around the optimum.
\end{eulercomment}
\begin{eulerprompt}
>writeln(povintersection([povsphere(x,0.5),povplane(c,c.x')], ...
>  povlook(red,0.9)));
\end{eulerprompt}
\begin{eulercomment}
And an error in the direction of the optimum.
\end{eulercomment}
\begin{eulerprompt}
>writeln(povarrow(x,c*0.5,povlook(red)));
\end{eulerprompt}
\begin{eulercomment}
We add text to the screen. Text is just a 3D object. We need to place
and turn it according to our view.
\end{eulercomment}
\begin{eulerprompt}
>writeln(povtext("Linear Problem",[0,0.2,1.3],size=0.05,rotate=5°)); ...
>povend();
\end{eulerprompt}
\eulerheading{Contoh Lainnya}
\begin{eulercomment}
Anda dapat menemukan beberapa contoh Povray di Euler di file berikut.


ee: Examples/Dandelin Spheres\\
See: Examples/Donat Math\\
See: Examples/Trefoil Knot\\
See: Examples/Optimization by Affine Scaling

\end{eulercomment}
\eulersubheading{Latihan Soal}
\begin{eulerprompt}
>plot3d("y^2+x^2",a=0,b=4,c=0,d=4,n=200,level=0,light=[1,1,0]):
\end{eulerprompt}
\eulerimg{27}{images/EMT4Plot3D_Saphira-058.png}
\begin{eulerprompt}
>plot3d("x^2+y^3",angle=0°,>contour,>spectral):
\end{eulerprompt}
\eulerimg{27}{images/EMT4Plot3D_Saphira-059.png}
\begin{eulerprompt}
>plot3d("x-(1/2x^3)-(1/4y^2)",r=1,title="z=x-(1/2x^3)-(1/4y^2)"):
\end{eulerprompt}
\eulerimg{27}{images/EMT4Plot3D_Saphira-060.png}
\begin{eulerprompt}
>plot3d("y^2+x^2",angle=0°,>contour,>spectral):
\end{eulerprompt}
\eulerimg{13}{images/EMT4Plot3D_Saphira-061.png}
\begin{eulerprompt}
>plot3d("x^3-y^4",>contour,angle=60°,level=-2:0.2:2):
\end{eulerprompt}
\eulerimg{13}{images/EMT4Plot3D_Saphira-062.png}
\end{eulernotebook}
\end{document}
