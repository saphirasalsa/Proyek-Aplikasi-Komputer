\documentclass{article}

\usepackage{eumat}

\begin{document}
\begin{eulernotebook}
\eulersubheading{Plot2D}
\begin{eulercomment}
Nama  : Saphira Nuria Salsabila\\
NIM   : 22305141050\\
Kelas : Matematika B

\end{eulercomment}
\eulersubheading{}
\eulersubheading{Subtopik 5 - Menggambar Beberapa Kurva Sekaligus}
\begin{eulercomment}
Dalam subtopik ini, kita akan membahas cara menggambar menggambar
kurva sekaligus (dalam satu baris perintah). Dengan adanya materi pada
subtopik ini, diharapkan kita bisa menggambar beberapa kurva dalam
bidang koordinat yang berbeda dalam satu waktu. Untuk melakukannya,
kita dapat menggunakan perintah figure(). Berikut contoh menggambar
beberapa kurva sekaligus.

Gambarkan plot fungsi berikut:\\
\end{eulercomment}
\begin{eulerformula}
\[
x^n, 1 \leq n \leq 4
\]
\end{eulerformula}
\begin{eulerprompt}
>reset; 
>figure(2,2); ...
>for n=1 to 4; figure(n); plot2d("x^"+n); end;...
>figure(0):
\end{eulerprompt}
\eulerimg{27}{images/Materi Sendiri_Saphira_22305141050-002.png}
\begin{eulercomment}
Penjelasan dari plot fungsi\\
\end{eulercomment}
\begin{eulerformula}
\[
x^n, 1 \leq n \leq 4
\]
\end{eulerformula}
\begin{eulercomment}
- reset;\\
Perintah ini digunakan sebagai awalan/permulaan menggambar grafik
baru. Perintah ini juga berfungsi untuk menghapus grafik yang telah
ada sebelumnya sehingga dapat memulai kembali untuk membuat grafik
yang baru.\\
- figure(2x2);\\
Perintah figure() digunakan untuk membuat jendela grafik dengan ukuran
axb di mana a merupakan banyaknya baris dan b merupakan banyaknya
kolom. Dalam hal ini perintah figure(2,2) berarti akan dibuat jendela
grafik berukuran 2x2 (2 baris dan 2 kolom).\\
- for n=1 to 4;\\
Perintah ini berfungsi untuk melakukan looping, dalam hal ini sebanyak
4 kali yaitu dari 1 sampai 4.\\
- figure(n);\\
Perintah ini berfungsi untuk beralih dari grafik satu ke grafik
lainnya (grafik ke-n)\\
- plot2d("x\textasciicircum{}"+n);\\
Perintah plot2d() berfungsi untuk membuat plot fungsi matematika.
Dalam hal ini fungsi yang diplot adalah x\textasciicircum{}n, di mana n adalah nilai
dari variabel yang dikenai looping. Sehingga, akan ditampilkan plot
dari x\textasciicircum{}1, x\textasciicircum{}2, x\textasciicircum{}3, dan x\textasciicircum{}4 pada bidang grafik.\\
- end;\\
Perintah ini menunjukkan berakhirnya looping;\\
- figure(0);\\
Perintah ini berfungsi untuk beralih kembali ke jendela grafik utama.

\end{eulercomment}
\begin{eulerprompt}
>figure(2,2); ...
>for n=1 to 4; figure(n); plot2d(x^n); end; 
\end{eulerprompt}
\begin{euleroutput}
  Variable x not found!
  Error in ^
  Error in:
  figure(2,2); for n=1 to 4; figure(n); plot2d(x^n); end;  ...
                                                  ^
\end{euleroutput}
\begin{eulercomment}
Dari masalah di atas, kita tahu bahwa untuk membuat kurva fungsi x\textasciicircum{}n
tidak dapat semerta-merta langsung diketik x\textasciicircum{}n pada baris perintah,
melainkan harus diketik dengan ("x\textasciicircum{}"+n). Tanda petik dua ("...")
berfungsi untuk mengidentifikasi ekspresi matematika. Sedangkan tanda
plus (+) berfungsi untuk menggabungkan string dengan variabel yang
ada.

\end{eulercomment}
\eulersubheading{}
\begin{eulercomment}
Contoh Soal :\\
Gambarkan plot fungsi\\
\end{eulercomment}
\begin{eulerformula}
\[
f(x)= 3x^3 - 5x; -2 \leq x \leq 2
\]
\end{eulerformula}
\begin{eulerprompt}
>reset;
>figure(3,3);...
>for k=1:9 figure(k); plot2d("3x^3-5x",-2,2,grid=k); end;...
>figure(0):
\end{eulerprompt}
\eulerimg{27}{images/Materi Sendiri_Saphira_22305141050-005.png}
\begin{eulercomment}
Penjelasan dari plot fungsi\\
\end{eulercomment}
\begin{eulerformula}
\[
f(x)= 3x^3-5x, -2 \leq x \leq 2
\]
\end{eulerformula}
\begin{eulercomment}
- reset;\\
Perintah ini berfungsi untuk memulai ulang perintah, dengan menghapus
grafik yang telah ada sebelumnya dan membuat perintah baru.\\
- figure(3,3);\\
Perintah ini berfungsi untuk membuat jendela grafik dengan ukuran 3x3.
Sehingga nantinya akan ada empat macam jendela grafik yang akan
ditampilkan dengan tata letak 3 baris dan 3 kolom.\\
- for k=1:9;\\
Perintah ini berfungsi untuk melakukan looping perintah sebanyak 9
kali.\\
- figure(k);\\
Perintah ini berfungsi untuk beralih dari jendela grafik satu ke
jendela grafik lainnya (grafik ke-k).\\
- plot2d("3x\textasciicircum{}3-5x,-2,2,grid=k");\\
Perintah plot2d() berfungsi untuk membuat plot fungsi matematika.
Dalam hal ini fungsi yang diplot adalah 3x\textasciicircum{}3-5x, dengan batas sumbu x
dari -2 sampai 2. Argumen grid=k berfungsi untuk mengaktifkan grid
pada jendela grafik ke-k.\\
- end;\\
Perintah ini menunjukkan berakhirnya proses looping\\
- figure(0);\\
Perintah ini berfungsi untuk kembali ke jendela grafik utama.

Dari contoh di atas dapat kita perhatikan bahwa tampilan plot dari
yang ke-1 hingga ke-9 memiliki tampilan yang berbeda-beda. EMT
memiliki berbagai gaya plot 2D yang dapat dijalankan menggunakan
perintah grid=n di mana n adalah jumlah langkah minimal. Setiap nilai
n memiliki tampilan plot adaptif yang berbeda dalam plot 2D, di
antaranya yaitu:\\
0 : tidak ada grid (kisi), frame, sumbu, dan label, hanya kurva saja\\
1 : dengan sumbu, label-label sumbu di luar frame jendela grafik\\
2 : tampilan default\\
3 : dengan grid pada sumbu x dan y, label-label sumbu berada di dalam
jendela grafik\\
4 : tidak ada grid (kisi), sumbu x dan y, dan label berada di luar
frame jendela grafik\\
5 : tampilan default tanpa margin di sekitar plot\\
6 : hanya dengan sumbu x y dan label, tanpa grid\\
7 : hanya dengan sumbu x y dan tanda-tanda pada sumbu.\\
8 : hanya dengan sumbu dan tanda-tanda pada sumbu, dengan tanda-tanda\\
yang lebih halus pada sumbu.\\
9 : tampilan default dengan tanda-tanda kecil di dalam jendela\\
10: hanya dengan sumbu-sumbu, tanpa tanda

\end{eulercomment}
\eulersubheading{}
\begin{eulercomment}
Contoh Soal:\\
Gambarkan plot fungsi berikut :\\
\end{eulercomment}
\begin{eulerformula}
\[
g(x)= 2x^3 - 3x
\]
\end{eulerformula}
\begin{eulerprompt}
>reset;
>aspect(); 
>figure(3,4);
>figure(1); plot2d("2x^3-3x", grid=0);... 
>figure(2); plot2d("2x^3-3x", grid=1);...  
>figure(3); plot2d("2x^3-3x", grid=2);... 
>figure(4); plot2d("2x^3-3x", grid=3);... 
>figure(5); plot2d("2x^3-3x", grid=4);... 
>figure(6); plot2d("2x^3-3x", grid=5);... 
>figure(7); plot2d("2x^3-3x", grid=6);... 
>figure(8); plot2d("2x^3-3x", grid=7);... 
>figure(9); plot2d("2x^3-3x", grid=8);... 
>figure(10); plot2d("2x^3-3x", grid=9);... 
>figure(11); plot2d("2x^3-3x", grid=10);... 
>figure(0):
\end{eulerprompt}
\eulerimg{27}{images/Materi Sendiri_Saphira_22305141050-008.png}
\begin{eulercomment}
Penjelasan dari plot fungsi:\\
\end{eulercomment}
\begin{eulerformula}
\[
g(x): 2x^3-3x
\]
\end{eulerformula}
\begin{eulercomment}
- aspect(1, 2);\\
Perintah aspect() berfungsi untuk mengatur rasio aspek dari jendela
grafik. Dalam hal ini, perintah aspect(1, 2); akan menghasilkan plot
dengan perbandingan panjang sumbu x dan sumbu y berturut-turut 1:2.\\
- figure (3,4);\\
Perintah ini digunakan untuk membuat jendela grafik dengan ukuran 3x4
(3 baris dan 4 kolom).\\
- figure(1); plot2d("2x\textasciicircum{}3-3x", grid(0);...\\
Merupakan perintah untuk beralih ke jendela grafik ke-1 dan menggambar
plot dari fungsi 2x\textasciicircum{}3-3x tanpa grid, frame, maupun sumbu.\\
- figure(2); plot2d("2x\textasciicircum{}3-3x",grid=1); ...\\
Merupakan perintah untuk beralih ke jendela grafik kedua dan
menggambar plot dari fungsi 2x\textasciicircum{}3-3x dengan grid hanya pada sumbu x dan
y.\\
- figure(3); plot2d("2x\textasciicircum{}3-3x",grid=2); ...\\
Merupakan perintah untuk beralih ke jendela grafik ketiga dan
menggambar plot dari fungsi 2x\textasciicircum{}3-3x dengan tampilan default, termasuk
tanda-tanda default pada sumbu.\\
- figure(4); plot2d("2x\textasciicircum{}3-3x",grid=3); ...\\
Merupakan perintah untuk beralih ke jendela grafik keempat dan
menggambar plot dari fungsi 2x\textasciicircum{}3-3x dengan grid pada sumbu x dan y,
serta label-label sumbu yang ada di dalam jendela.\\
- figure(5); plot2d("2x\textasciicircum{}3-3x",grid=4); ...\\
Merupakan perintah untuk beralih ke jendela grafik kelima dan
menggambar plot dari fungsi 2x\textasciicircum{}3-3x tanpa tanda-tanda sumbu, hanya
label-label yang ada.\\
- figure(6); plot2d("2x\textasciicircum{}3-3x",grid=5); ...\\
Merupakan perintah untuk beralih ke jendela grafik keenam dan
menggambar plot dari fungsi 2x\textasciicircum{}3-3x dengan tampilan default, tetapi
tanpa margin di sekitar plot.\\
- figure(7); plot2d("2x\textasciicircum{}3-3x",grid=6); ...\\
Merupakan perintah untuk beralih ke jendela grafik ketujuh dan
menggambar plot dari fungsi 2x\textasciicircum{}3-3x hanya dengan sumbu-sumbu (tanpa
grid atau label).\\
- figure(8); plot2d("2x\textasciicircum{}3-3x",grid=7); ...\\
Merupakan perintah untuk beralih ke jendela grafik kedelapan dan
menggambar plot dari fungsi 2x\textasciicircum{}3-3x hanya dengan sumbu-sumbu dan
tanda-tanda pada sumbu.\\
- figure(9); plot2d("2x\textasciicircum{}3-3x",grid=8); ...\\
Merupakan perintah untuk beralih ke jendela grafik kesembilan dan
menggambar plot dari fungsi 2x\textasciicircum{}3-3x hanya dengan sumbu-sumbu dan
tanda-tanda pada sumbu, dengan tanda-tanda yang lebih halus pada
sumbu.\\
- figure(10); plot2d("2x\textasciicircum{}3-3x",grid=9); ...\\
Merupakan perintah untuk beralih ke jendela grafik kesepuluh dan
menggambar plot dari fungsi 2x\textasciicircum{}3-3x dengan tanda-tanda default kecil
di dalam jendela.\\
- figure(11); plot2d("2x\textasciicircum{}3-3x",grid=10); ...\\
Merupakan perintah untuk beralih ke jendela grafik kesebelas dan
menggambar plot dari fungsi 2x\textasciicircum{}3-3x hanya dengan sumbu-sumbu, tanpa
tanda-tanda.\\
- figure(0);\\
Merupakan perintah untuk beralih kembali ke jendela grafik utama atau
jendela grafik dengan nomor 0 setelah semua perintah dalam urutan
selesai dieksekusi.

Dari ketiga contoh di atas, dapat disimpulkan bahwa untuk menggambar
beberapa kurva sekaligus, dapat dilakukan dengan satu baris perintah
ataupun dengan cara mendefinisikan 1 per 1.

\end{eulercomment}
\eulersubheading{}
\begin{eulercomment}
Terlihat beberapa jenis grid memiliki tampilan yang mirip atau sama,
seperti 1 dan 2, 2 dan 5, 4 dan 9, serta 7, dan 8. Untuk dapat
membedakannya secara lebih jelas, kita akan mengubah grid dari contoh
berikut.
\end{eulercomment}
\begin{eulerprompt}
>reset;
>aspect(2,3);
>figure(2,1);...
>figure(1); plot2d("sin(x)",0,4pi, grid=2);...
>figure(2); plot2d("sin(2x)",0,4pi, grid=2,color=red,style="--",);
>figure(0):
\end{eulerprompt}
\eulerimg{34}{images/Materi Sendiri_Saphira_22305141050-010.png}
\begin{eulercomment}
Dalam visualisasi plot, dapat dilakukan berbagai modifikasi mulai dari
jenis grid, warna, style, dan lain-lain. Modifikasi tersebut akan
dibahas lebih lanjut pada subtopik selanjutnya.\\
\end{eulercomment}
\eulersubheading{}
\begin{eulercomment}
Contoh Soal:

Gambarkan plot fungsi-fungsi berikut:\\
\end{eulercomment}
\begin{eulerformula}
\[
f(x) = x^2 e^{-x}, 0 \leq x \leq 10
\]
\end{eulerformula}
\begin{eulerformula}
\[
g(x) = 2 e^{x}, -5 \leq x \leq 5
\]
\end{eulerformula}
\begin{eulerformula}
\[
h(x) = e^{x^2}, -2 \leq x \leq 2
\]
\end{eulerformula}
\begin{eulerprompt}
>reset;
>aspect(3,1);
>figure(1,3);...
>figure(1); plot2d("x^2*exp(-x)",0,10);...
>figure(2); plot2d("2*exp(x)",-5,5);...
>figure(3); plot2d("exp(x^2)",-2,2);...
>figure(0):
\end{eulerprompt}
\eulerimg{8}{images/Materi Sendiri_Saphira_22305141050-014.png}
\eulersubheading{}
\begin{eulercomment}
Contoh Soal:

Gambarkan plot dari fungsi berikut:\\
\end{eulercomment}
\begin{eulerformula}
\[
2xlog(x^2)
\]
\end{eulerformula}
\begin{eulerprompt}
>reset;
>aspect(3/2); 
>figure(1,2);...
>for a=1:2 ;figure(a); plot2d("2*x*log(x^2)",-2,2,grid=a); end;...  
>figure(0):
\end{eulerprompt}
\eulerimg{17}{images/Materi Sendiri_Saphira_22305141050-016.png}
\begin{eulercomment}
\end{eulercomment}
\eulersubheading{}
\eulersubheading{Subtopik 14 - Menggambar Daerah yang Dibatasi Beberapa Kurva}
\begin{eulercomment}
Pada subtopik sebelumnya telah kita ketahui dan pelajari bersama bahwa
EMT dapat melakukan visualisasi plot mulai dari bentuk ekspresi
langsung hingga plot dari fungsi-fungsi.\\
Subtopik ini merupakan kelanjutan dari subtopik sebelumnya, yaitu
membentuk/menggambar daerah dari perpotongan beberapa kurva yang telah
didefinisikan. Hal ini dapat bermanfaat untuk membantu dalam
menyelesaikan permasalahan dalam matematika, salah satu contohnya
seperti optimasi program linear, di mana disajikan beberapa
fungsi-fungsi kendala beserta dengan fungsi tujuannya dan perlu
divisualisasikan dalam bentuk grafik untuk melihat dimana letak daerah
layaknya untuk menentukan nilai optimum.

Dalam EMT ada beberapa perintah yang digunakan untuk menggambar daerah
yang dibatasi oleh beberapa kurva, di antaranya yaitu:\\
- plot2d\\
Digunakan untuk melakukan plotting.\\
- filled=true\\
Digunakan untuk memberikan isian/arsiran pada daerah/area di bawah
kurva saat plotting.\\
- style=”...”\\
Digunakan untuk memilih gaya kurva yang akan digunakan saat plotting.
Anda dapat memilih dari beberapa gaya, seperti ”#”, ”/”, ”\textbackslash{}”, atau
”-”. Dan hal ini mempengaruhi tampilan daerah kurva yang terbentuk.\\
- fillcolor\\
Digunakan untuk menentukan warna isian yang akan digunakan untuk\\
mengiri area di bawah kurva.
\end{eulercomment}
\begin{eulerprompt}
>t=linspace(0, 2pi, 1000); //parameter untuk kurva
>x=cos(t)*exp(t/pi); y=sin(t/pi); //x(t) dan y(t)
>figure(1,2); aspect(3/2)
>figure(1); plot2d(x,y,r=7); //plot kurva
>figure(2); plot2d(x,y,r=7,>filled,style="/",fillcolor=blue); //mengisi kurva
>figure(0):
\end{eulerprompt}
\eulerimg{17}{images/Materi Sendiri_Saphira_22305141050-017.png}
\begin{eulercomment}
Penjelasan\\
- t=linspace(0,2pi,1000);\\
Pada langkah pertama yaitu mendefinisikan parameter t sebagai
serangkaian 1000 titik antara 0 dan 2pi. Parameter t ini akan
digunakan sebagai parameter untuk menggambar kurva.

- cos(t)*exp(t/pi); y=sin(t)*exp(t/pi);\\
Didefiniskan dua vektor x dan y yang merupakan koordinat x dan y dari
kurva yang akan digambar.\\
Fungsi :\\
\end{eulercomment}
\begin{eulerformula}
\[
cos(t)*exp(t/pi)
\]
\end{eulerformula}
\begin{eulercomment}
digunakan untuk menghitung komponen x(x(t)), dan\\
\end{eulercomment}
\begin{eulerformula}
\[
sin(t)*exp(t/pi)
\]
\end{eulerformula}
\begin{eulercomment}
digunakan untuk menghitung komponen y (y(t)) dari kurva.

- figure(1,2); aspect(3/2)\\
Perintah ini digunakan untuk mengatur tampilan gambar. Perintah
figure(1,2) digunakan membuat dua gambar (1 dan 2) dalam satu jendela
gambar. Dan perintah aspect(3/2) mengatur rasio aspek gambar\\
menjadi 3:2, yang mempengaruhi bentuk dan ukuran gambar yang akan
digambar.

- figure(1); plot2d(x,y,r=7);\\
Perintah ini memilih gambar pertama (1) dan menggunakan perintah
plot2d untuk menggambar kurva yang dihitung sebelumnya. Parameter r=10
mengatur lebar garis plot. Ini menghasilkan kurva tanpa adanya isi
atau arsiran di dalamnya.

- figure(2); plot2d(x,y,r=7,\textgreater{}filled,style=”/”,fillcolor=blue);\\
Selanjutnya pada perintah ini beralih ke gambar kedua (2) dan
menggunakan perintah plot2d lagi untuk menggambar kurva yang sama
dengan pengisian area di bawahnya. Perintah \textgreater{}filled digunakan untuk\\
mengisi area di bawah kurva, style=”/” digunakan untuk mengatur gaya
garis menjadi garis miring, dan fillcolor=blue digunakan untuk
mengatur warna isian menjadi biru.

- figure(0);\\
Baris perintah ini digunakan untuk mengakhiri gambar dan kembali ke
tampilan biasa tanpa gambar. Perintah ini berfungsi untuk
menyelesaikan proses penggambaran.
\end{eulercomment}
\begin{eulerprompt}
>x=linspace(0,2pi,100); plot2d(cos(x),sin(x)*0.5,r=1,>filled,style="\(\backslash\)"):
\end{eulerprompt}
\eulerimg{17}{images/Materi Sendiri_Saphira_22305141050-020.png}
\begin{eulercomment}
Penjelasan:\\
- x=linspace(0,2pi,100);\\
Mendefinisikan vektor x dengan menggunakan perintah linspace. linspace
digunakan untuk membuat vektor dengan 100 titik yang secara merata
tersebar antara 0 dan 2phi. Dalam konteks ini, vektor x akan digunakan
sebagai parameter saat menggambar kurva.

- plot2d(cos(x),sin(x)*0.5,r=1,\textgreater{}filled,style=”\textbackslash{}”):\\
Ini merupakan perintah utama yang digunakan untuk menggambar plot.
Perintah ini memiliki beberapa parameter sebagai berikut:\\
\textgreater{} cos(x) adalah komponen x dari kurva. Ini adalah hasil dari fungsi
kosinus yang dihitung pada vektor x.\\
\textgreater{} sin(x)*0.5 adalah komponen y dari kurva. Ini adalah hasil dari
fungsi sinus yang dihitung pada vektor x dan kemudian dikalikan dengan
0,5, yang mengubah amplitudonya.\\
\textgreater{} r=1 mengatur lebar garis plot menjadi 1.\\
\textgreater{} filled digunakan untuk mengisi area di bawah kurva, sehingga
menciptakan daerah yang terisi.\\
\textgreater{} style=”\textbackslash{}” mengatur gaya garis kurva menjadi garis miring kekiri
dengan menggunakan tanda backslash(\textbackslash{})
\end{eulercomment}
\begin{eulerprompt}
>t=linspace(0,2pi,4);
>plot2d(cos(t),sin(t),>filled, style="/",fillcolor=orange,r=1.5):
\end{eulerprompt}
\eulerimg{17}{images/Materi Sendiri_Saphira_22305141050-021.png}
\begin{eulercomment}
Penjelasan:

- t=linspace(0,2pi,4);\\
Pada perintah ini, kita definisikan vektor t dengan menggunakan
perintah linspace. Linspace digunakan untuk membuat vektor dengan 4
titik yang terletak secara merata antara 0 dan 2pi. Dalam konteks ini,
vektor t akan digunakan sebagai parameter saat menggambar kurva.

- plot2d(cos(t),sin(t),\textgreater{}filled,style=”/”,fillcolor=orange,r=1.2):\\
Perntah ini merupakan perintah utama yang digunakan untuk menggambar
plot.\\
Perintah ini memiliki beberapa parameter sebagai berikut:\\
\textgreater{} cos(t) adalah komponen x dari kurva.\\
\textgreater{} sin(t) adalah komponen y dari kurva.\\
\textgreater{} filled digunakan untuk mengisi area di bawah kurva, sehingga
menciptakan bentuk yang terisi. Ini\\
berarti daerah di bawah kurva akan diwarnai.\\
\textgreater{} style=”/” mengatur gaya garis kurva menjadi garis miring (”/”).\\
\textgreater{} fillcolor=orange mengatur warna isian daerah di bawah kurva menjadi
oranye.\\
\textgreater{} r=1.2 mengatur lebar garis plot menjadi 1.2.
\end{eulercomment}
\begin{eulerprompt}
>t=linspace(0,2pi,6); plot2d(cos(t),sin(t),>filled,style="#"):
\end{eulerprompt}
\eulerimg{17}{images/Materi Sendiri_Saphira_22305141050-022.png}
\begin{eulercomment}
Penjelasan:\\
- t=linspace(0,2pi,6);\\
Pada perintah ini, kita definisikan vektor t dengan menggunakan
perintah linspace. Linspace digunakan untuk membuat vektor dengan 6
titik yang terletak secara merata antara 0 dan 2pi. Dalam konteks ini,
vektor t akan digunakan sebagai parameter saat menggambar kurva.

- plot2d(cos(t),sin(t),\textgreater{}filled,style=””):\\
Ini adalah perintah utama yang digunakan untuk menggambar plot.
Perintah ini memiliki beberapa parameter sebagai berikut:\\
\textgreater{} cos(t) adalah komponen x dari kurva.\\
\textgreater{} sin(t) adalah komponen y dari kurva.\\
\textgreater{} filled digunakan untuk mengisi area di bawah kurva, sehingga
menciptakan bentuk yang terisi. Ini berarti daerah di bawah kurva akan
diisi dengan warna atau pola tertentu.\\
\textgreater{} style=”#” mengatur isian kurva menjadi warna solid dengan
menggunakan simbol tanda pagar (”#”)
\end{eulercomment}
\begin{eulerprompt}
>t=linspace(0,2pi,7); ...
>plot2d(sin(t),cos(t),r=1,>filled,style="/",fillcolor=blue):
\end{eulerprompt}
\eulerimg{17}{images/Materi Sendiri_Saphira_22305141050-023.png}
\begin{eulercomment}
Penjelasan:

- t=linspace(0,2pi,7);:\\
Fungsi linspace digunakan untuk membuat array berisi sejumlah nilai
yang merata dalam rentang tertentu. Dalam hal ini, rentangnya adalah
dari 0 hingga 2pi (dua kali nilai pi) dan sebanyak 7 titik akan
dihasilkan. Ini akan digunakan sebagai sudut dalam koordinat polar
untuk menggambarkan data.

- plot2d(sin(t),cos(t),r=1,\textgreater{}filled,style=”/”,fillcolor=blue):\\
Ini adalah perintah untuk melakukan plotting data. Terdapat beberapa
argumen di sini:\\
\textgreater{} sin(t): Ini adalah nilai sinus dari setiap elemen dalam array t. Ini
akan digunakan sebagai komponen sumbu X dalam koordinat polar.\\
\textgreater{} cos(t): Ini adalah nilai kosinus dari setiap elemen dalam array t.
Ini akan digunakan sebagai komponen sumbu Y dalam koordinat polar.\\
\textgreater{} r=1: Ini adalah argumen opsional yang menentukan radius plot. Dalam
hal ini, radiusnya diatur menjadi 1.\\
\textgreater{} filled: Ini adalah argumen yang menginstruksikan untuk mengisi area
di dalam kurva plot.\\
\textgreater{} style=”/”: Ini adalah argumen yang menentukan gaya garis yang
digunakan untuk plot. Di sini, garisnya akan berbentuk garis miring
(”/”).\\
\textgreater{} fillcolor=blue: Ini adalah argumen yang menentukan warna pengisian
untuk area di dalam kurva plot. Dalam hal ini, warnanya diatur menjadi
biru.
\end{eulercomment}
\begin{eulerprompt}
>t=linspace(0,2pi,1000); x=cos(3*t); y=sin(4*t);
>plot2d(x,y,<grid,<frame,>filled):
\end{eulerprompt}
\eulerimg{17}{images/Materi Sendiri_Saphira_22305141050-024.png}
\begin{eulercomment}
Penjelasan:

- t = linspace(0, 2*pi, 1000);\\
Ini adalah perintah untuk membuat vektor t yang berisi 1000 nilai yang
merata terdistribusi antara 0 hingga 2pi. Vektor t ini akan digunakan
sebagai parameter waktu atau sudut dalam parameterisasi lingkaran.\\
linspace(0, 2*pi, 1000) membuat 1000 titik antara 0 hingga 2pi,
memberikan sudut-sudut yang merata di sepanjang satu putaran
lingkaran.

- x = cos(3*t); y = sin(4*t);\\
Ini adalah perintah untuk menghitung vektor x dan y yang menggambarkan
lintasan dalam koordinat polar.\\
\textgreater{} x = cos(3*t); menghitung nilai x sebagai hasil dari fungsi kosinus
dari 3 kali nilai t. Ini akan menghasilkan osilasi yang lebih cepat
pada sumbu x.\\
\textgreater{} y = sin(4*t); menghitung nilai y sebagai hasil dari fungsi sinus
dari 4 kali nilai t. Ini akan menghasilkan osilasi yang lebih cepat
pada sumbu y.

- plot2d(x, y, \textless{}grid, \textless{}frame, \textgreater{}filled);\\
Ini adalah perintah untuk membuat plot dari vektor x dan y. Berikut
adalah rincian perintah ini:\\
x adalah vektor yang digunakan sebagai data untuk sumbu x.\\
y adalah vektor yang digunakan sebagai data untuk sumbu y.\\
\textless{}grid mengaktifkan garis-garis koordinat (grid) di latar belakang
plot, membantu dalam visualisasi.\\
\textless{}frame mengaktifkan bingkai (frame) di sekitar plot.\\
\textgreater{}filled mengisi area di bawah kurva dengan warna, membuat plot menjadi
lebih berwarna
\end{eulercomment}
\begin{eulerprompt}
>t=0:0.1:1; ...  
>plot2d(t,interval(t-random(size(t)),t+random(size(t))),style="|"); ...
>plot2d(t,t,add=true):
\end{eulerprompt}
\eulerimg{17}{images/Materi Sendiri_Saphira_22305141050-025.png}
\begin{eulercomment}
Penjelasan:

- t = 0:0.1:1;\\
Ini adalah perintah untuk membuat vektor t yang berisi nilai-nilai
dari 0 hingga 1 dengan interval 0.1.\\
Hasilnya adalah vektor [0, 0.1, 0.2, 0.3, ..., 0.9, 1].

- plot2d(t, interval(t - random(size(t)), t + random(size(t))),
style=”\textbar{}”);\\
Ini adalah perintah untuk membuat plot pertama. Rincian perintah ini
adalah sebagai berikut:\\
\textgreater{} interval(t - random(size(t)), t + random(size(t))) adalah interval
yang digunakan untuk menggambar\\
”garis” pada plot. Setiap titik pada sumbu x (t) akan dihubungkan oleh
dua garis vertikal yang dibuat secara acak di sekitar titik tersebut
menggunakan random(size(t)). Hasilnya adalah plot dengan garisgaris
vertikal yang mewakili interval acak di sekitar setiap titik pada
sumbu x.\\
\textgreater{} style=”\textbar{}” mengatur gaya plot menjadi garis vertikal (”\textbar{}”).

- plot2d(t, t, add=true);\\
Ini adalah perintah untuk membuat plot kedua dan menambahkannya ke
dalam plot yang sudah ada dari perintah sebelumnya. Rincian perintah
ini adalah sebagai berikut:\\
\textgreater{} t adalah sumbu x dan y plot ini, sehingga plot ini akan menjadi plot
garis diagonal dengan kemiringan 45 derajat.\\
\textgreater{} add=true digunakan untuk menambahkan plot ini ke dalam plot
sebelumnya, sehingga kedua plot akan ditampilkan dalam satu plot yang
sama.
\end{eulercomment}
\begin{eulerprompt}
>t=-1:0.01:1; x=~t-0.01,t+0.01~; y=x^3-x;
>plot2d(t,y):
\end{eulerprompt}
\eulerimg{17}{images/Materi Sendiri_Saphira_22305141050-026.png}
\begin{eulercomment}
Penjelasan:

- t = -1:0.01:1;\\
Ini adalah perintah untuk membuat vektor t yang berisi nilai-nilai
dari -1 hingga 1 dengan interval 0.01.\\
Hasilnya adalah vektor t yang berisi nilai-nilai seperti [-1, -0.99,
-0.98, ..., 0.99, 1]. Vektor t ini akan digunakan sebagai sumbu x pada
plot.

- x = t - 0.01, t + 0.01 ;\\
Ini adalah perintah yang menghitung vektor x. Tanda digunakan di sini
untuk mendefinisikan dua interval, yaitu [ t - 0.01, t + 0.01 ]. Ini
menghasilkan vektor x yang memiliki dua interval, satu yang kurang
dari t - 0.01 dan satu yang lebih dari t + 0.01.

- y = xˆ3 - x;\\
Ini adalah perintah yang menghitung vektor y sebagai fungsi dari x.
Fungsi ini menghitung nilai y dengan memasukkan setiap nilai x ke
dalam rumus xˆ3 - x.

- plot2d(t, y);\\
Ini adalah perintah untuk membuat plot dari fungsi y sebagai fungsi
dari t. Rincian perintah ini adalah sebagai berikut:\\
\textgreater{} t adalah sumbu x pada plot, yang berisi vektor t yang telah
didefinisikan sebelumnya.\\
\textgreater{} y adalah sumbu y pada plot, yang berisi vektor y yang dihitung dari
rumus xˆ3 - x.
\end{eulercomment}
\begin{eulerprompt}
>expr := "2*x^2+x*y+3*y^4+y"; // mendefinisikan sebuah ekspresi f(x,y)
>plot2d(expr,level=[0;1],style="-",color=blue): // 0 <= f(x,y) <= 1
\end{eulerprompt}
\eulerimg{17}{images/Materi Sendiri_Saphira_22305141050-027.png}
\begin{eulercomment}
Penjelasan:

- expr := ”2*xˆ2+x*y+3*yˆ4+y”;\\
Ini adalah perintah untuk mendefinisikan ekspresi matematika yang
disimpan dalam variabel expr. Ekspresi ini merupakan suatu fungsi f(x,
y) yang tergantung pada dua variabel, yaitu x dan y. Ekspresi ini
memiliki bentuk matematika yang terdiri dari berbagai suku, seperti
kuadrat dari x, perkalian x*y, kuadrat dari y, dan lainnya.

- plot2d(expr, level=[0;1], style=”-”, color=blue);\\
Ini adalah perintah untuk membuat plot dari fungsi f(x, y) yang telah
didefinisikan sebelumnya. Berikut adalah rincian perintah ini:\\
\textgreater{} expr adalah ekspresi yang akan digunakan sebagai fungsi yang akan
diplotkan. Dalam hal ini, ekspresi 2*xˆ2+x*y+3*yˆ4+y adalah fungsi
f(x, y) yang telah didefinisikan sebelumnya.\\
\textgreater{} level=[0;1] mengatur tingkat kontur (contour levels) yang akan
digunakan dalam plot. Dalam hal ini, tingkat kontur berada pada
interval 0 hingga 1, yang berarti plot akan menunjukkan wilayah di
mana f(x, y) memiliki nilai antara 0 hingga 1.\\
\textgreater{} style=”-” mengatur gaya plot menjadi garis berjenis -, yang akan
menghasilkan plot kontur.\\
\textgreater{} color=blue mengatur warna garis plot menjadi biru

Kita juga dapat mengisi rentang nilai seperti berikut:\\
\end{eulercomment}
\begin{eulerformula}
\[
-1\leq(x^2+y^2)^2-x^2+y^2\leq 0 
\]
\end{eulerformula}
\begin{eulerprompt}
>plot2d("(x^2+y^2)^2-x^2+y^2",r=1.2,level=[-1;0],style="/\(\backslash\)"):
\end{eulerprompt}
\eulerimg{17}{images/Materi Sendiri_Saphira_22305141050-029.png}
\begin{eulercomment}
Penjelasan:\\
- plot2d(”(xˆ2+yˆ2)ˆ2-xˆ2+yˆ2”, r=1.2, level=[-1;0], style=”/\textbackslash{}”);\\
Ini adalah perintah untuk membuat plot dari fungsi matematika yang
didefinisikan dalam bentuk string:\\
”(xˆ2+yˆ2)ˆ2-xˆ2+yˆ2”. Fungsi ini tergantung pada dua variabel, yaitu
x dan y.\\
(xˆ2+yˆ2)ˆ2-xˆ2+yˆ2 adalah rumus dari fungsi matematika yang akan
diplotkan.\\
- r=1.2 mengatur rentang (range) plot untuk kedua sumbu x dan y. Dalam
hal ini, rentangnya adalah [-1.2, 1.2], yang berarti plot akan berada
dalam wilayah ini.\\
- level=[-1;0] mengatur tingkat kontur (contour levels) yang akan
digunakan dalam plot. Dalam hal ini, ada dua tingkat kontur: -1 dan 0.
Ini akan menentukan wilayah kontur dalam plot.\\
- style=”/\textbackslash{}” mengatur gaya plot menjadi garis berpotongan (”/\textbackslash{}”). \\
Ini akan menghasilkan plot dengan garis-garis berpotongan yang
menggambarkan kontur fungsi. Di sini menggunakan kombinasi dua simbol
yaitu slash dan backslash ("/\textbackslash{}") untuk membuat garis-garis yang
berpotongan.
\end{eulercomment}
\begin{eulerprompt}
>plot2d("sin(x)^3","cos(x)",xmin=0,xmax=2pi,>filled,style="|"):
\end{eulerprompt}
\eulerimg{17}{images/Materi Sendiri_Saphira_22305141050-030.png}
\begin{eulercomment}
Penjelasan:

plot2d(”sin(x)ˆ3”, ”cos(x)”, xmin=0, xmax=2*pi, \textgreater{}filled, style=”\textbar{}”);\\
Ini adalah perintah untuk membuat plot dari dua fungsi matematika,
yaitu sin(x)ˆ3 dan cos(x), dalam satu plot yang sama. Berikut adalah
rincian perintah ini:

- ”sin(x)ˆ3” adalah ekspresi pertama yang akan diplotkan. Ini adalah
fungsi trigonometri sin(x) yang dipangkatkan tiga. Fungsi ini
tergantung pada variabel x.

- ”cos(x)” adalah ekspresi kedua yang akan diplotkan. Ini adalah
fungsi trigonometri cos(x). Fungsi ini juga tergantung pada variabel
x.

- xmin=0 dan xmax=2*pi mengatur rentang (range) plot untuk sumbu x
dari 0 hingga 2pi. Ini adalah rentang yang akan ditampilkan dalam
plot.

- \textgreater{}filled mengisi area di bawah kurva fungsi dengan warna, sehingga
area di bawah kurva fungsi akan diisi dengan warna.

- style=”\textbar{}” mengatur gaya plot menjadi garis vertikal (”\textbar{}”). Ini akan
menghasilkan plot dengan garis-garis vertikal.

\end{eulercomment}
\eulersubheading{Contoh-Contoh Soal}
\begin{eulercomment}
1. Gambarkan plot fungsi berikut ini\\
\end{eulercomment}
\begin{eulerformula}
\[
6x^2+5y^2+2xy+4x+3y
\]
\end{eulerformula}
\begin{eulercomment}
dengan interval\\
\end{eulercomment}
\begin{eulerformula}
\[
0 \leq f(x,y) \leq 1.5
\]
\end{eulerformula}
\begin{eulerprompt}
>expr := "6*x^2+5*y^2+2x*y+4x+3y"; // mendefinisikan sebuah ekspresi f(x,y)
>plot2d(expr,level=[0;1.5],style="-",color=red): // 0 <= f(x,y) <= 1.5
\end{eulerprompt}
\eulerimg{17}{images/Materi Sendiri_Saphira_22305141050-033.png}
\begin{eulercomment}
2. Gambarkan plot fungsi berikut:\\
\end{eulercomment}
\begin{eulerformula}
\[
2x^2+3x+1
\]
\end{eulerformula}
\begin{eulerformula}
\[
2x+6
\]
\end{eulerformula}
\begin{eulercomment}
dengan\\
\end{eulercomment}
\begin{eulerformula}
\[
xmin=0
\]
\end{eulerformula}
\begin{eulerformula}
\[
xmax=10
\]
\end{eulerformula}
\begin{eulerprompt}
>plot2d("2*x^2+3*x+1","2*x+6",xmin=0,xmax=10,>filled,fillcolor=blue,style="\(\backslash\)"):
\end{eulerprompt}
\eulerimg{17}{images/Materi Sendiri_Saphira_22305141050-038.png}
\begin{eulercomment}
3. Gambarkan plot fungsi berikut:\\
\end{eulercomment}
\begin{eulerformula}
\[
3x^2+3x+1
\]
\end{eulerformula}
\begin{eulerformula}
\[
2x+6
\]
\end{eulerformula}
\begin{eulerformula}
\[
-3x+2
\]
\end{eulerformula}
\begin{eulercomment}
dengan\\
\end{eulercomment}
\begin{eulerformula}
\[
xmin= -10
\]
\end{eulerformula}
\begin{eulerformula}
\[
xmax= 0
\]
\end{eulerformula}
\begin{eulerprompt}
>plot2d("3*x^2+3*x+1","2*x+6","-3*x+2",xmin=-10,xmax=0,>filled,fillcolor=orange,style="#"):
\end{eulerprompt}
\eulerimg{17}{images/Materi Sendiri_Saphira_22305141050-044.png}
\end{eulernotebook}
\end{document}
